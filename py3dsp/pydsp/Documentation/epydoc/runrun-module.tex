%
% API Documentation for API Documentation
% Module runrun
%
% Generated by epydoc 3.0alpha2
% [Thu Jul 20 16:44:54 2006]
%

%%%%%%%%%%%%%%%%%%%%%%%%%%%%%%%%%%%%%%%%%%%%%%%%%%%%%%%%%%%%%%%%%%%%%%%%%%%
%%                          Module Description                           %%
%%%%%%%%%%%%%%%%%%%%%%%%%%%%%%%%%%%%%%%%%%%%%%%%%%%%%%%%%%%%%%%%%%%%%%%%%%%

    \index{runrun \textit{(module)}|(}
\section{Module runrun}

    \label{runrun}
runrun.py, parallel file of runrun.fth.

runrun interfaces the run data of run.py with the hardware of dsp.py not to
mention the fits file, DV, and showall code. it is kinda the glue code that
brings things together


%%%%%%%%%%%%%%%%%%%%%%%%%%%%%%%%%%%%%%%%%%%%%%%%%%%%%%%%%%%%%%%%%%%%%%%%%%%
%%                               Functions                               %%
%%%%%%%%%%%%%%%%%%%%%%%%%%%%%%%%%%%%%%%%%%%%%%%%%%%%%%%%%%%%%%%%%%%%%%%%%%%

  \subsection{Functions}

    \label{runrun:burst}
    \index{runrun \textit{(module)}!runrun.burst \textit{(function)}}

    \vspace{0.5ex}

    \begin{boxedminipage}{\textwidth}

    \raggedright \textbf{burst}()

    \vspace{-1.5ex}

    \rule{\textwidth}{0.5\fboxrule}
    Turns off crun mode and returns to the normal acquisition mode.

    \vspace{1ex}

    \end{boxedminipage}

    \label{runrun:minItime}
    \index{runrun \textit{(module)}!runrun.minItime \textit{(function)}}

    \vspace{0.5ex}

    \begin{boxedminipage}{\textwidth}

    \raggedright \textbf{minItime}()

    \vspace{-1.5ex}

    \rule{\textwidth}{0.5\fboxrule}
    Computes and returns the minimum value of the itime. type "print 
    minItime()" to see the result at the command loop.

    \vspace{1ex}

    \end{boxedminipage}

    \label{runrun:scantime}
    \index{runrun \textit{(module)}!runrun.scantime \textit{(function)}}

    \vspace{0.5ex}

    \begin{boxedminipage}{\textwidth}

    \raggedright \textbf{scantime}()

    \vspace{-1.5ex}

    \rule{\textwidth}{0.5\fboxrule}
    Compute and return the total amount of time it will take to acquire an 
    image.

    \vspace{1ex}

    \end{boxedminipage}

    \label{runrun:src}
    \index{runrun \textit{(module)}!runrun.src \textit{(function)}}

    \vspace{0.5ex}

    \begin{boxedminipage}{\textwidth}

    \raggedright \textbf{src}()

    \vspace{-1.5ex}

    \rule{\textwidth}{0.5\fboxrule}
    Set a flag to make the current DV buffer store a source image. NOTE: 
    This function should not be used by the user directly, use only if you 
    know what you are doing. This function is called by the 'sscan' and 
    'srun' commands.

    \vspace{1ex}

    \end{boxedminipage}

    \label{runrun:bkg}
    \index{runrun \textit{(module)}!runrun.bkg \textit{(function)}}

    \vspace{0.5ex}

    \begin{boxedminipage}{\textwidth}

    \raggedright \textbf{bkg}()

    \vspace{-1.5ex}

    \rule{\textwidth}{0.5\fboxrule}
    Set a flag to make the current DV buffer store a background image. 
    NOTE: This function should not be used by the user directly, use only 
    if you know what you are doing. This function is called by the 'bscan' 
    and 'brun' commands.

    \vspace{1ex}

    \end{boxedminipage}

    \label{runrun:vdiode}
    \index{runrun \textit{(module)}!runrun.vdiode \textit{(function)}}

    \vspace{0.5ex}

    \begin{boxedminipage}{\textwidth}

    \raggedright \textbf{vdiode}(\textit{wfile}=\texttt{sys.stdout})

    \vspace{-1.5ex}

    \rule{\textwidth}{0.5\fboxrule}
    Print out the current diode voltage.

    \vspace{1ex}

    \end{boxedminipage}

    \label{runrun:preCheckItime}
    \index{runrun \textit{(module)}!runrun.preCheckItime \textit{(function)}}

    \vspace{0.5ex}

    \begin{boxedminipage}{\textwidth}

    \raggedright \textbf{preCheckItime}(\textit{raw\_input}=\texttt{{\textless}built-in function raw\_input{\textgreater}}, \textit{wfile}=\texttt{sys.stdout})

    \vspace{-1.5ex}

    \rule{\textwidth}{0.5\fboxrule}
    This calls the scantime command with current settings to see if the set
    itime is appropriate. If itime is too short, a warning is presented to 
    the user and waits for 3 possible response: * A carriage return, a 'n' 
    or 'N' in the input string which aborts back to the command loop. * A 
    'y' or 'Y' where the itime will be set to the minimum possible. * If 
    the user types a number, it will attempt to use that number instead as 
    a correction to the original input value.

    \vspace{1ex}

    \end{boxedminipage}

    \label{runrun:calibrateItime}
    \index{runrun \textit{(module)}!runrun.calibrateItime \textit{(function)}}

    \vspace{0.5ex}

    \begin{boxedminipage}{\textwidth}

    \raggedright \textbf{calibrateItime}()

    \vspace{-1.5ex}

    \rule{\textwidth}{0.5\fboxrule}
    A Finds the magic calibration constants.

    Does a small number of frames w many columns and few rows does a small 
    number of frames many rows and few columns does many frames with few 
    rows and few columns uses linear algebra to figure out solution. 
    modifies values for the rest of the run of pydsp. constants are 
    persisted by hand editing runrun.py. adding them to the run dict is an 
    exercise for the user. ;-)

    \vspace{1ex}

    \end{boxedminipage}

    \label{runrun:bscan}
    \index{runrun \textit{(module)}!runrun.bscan \textit{(function)}}

    \vspace{0.5ex}

    \begin{boxedminipage}{\textwidth}

    \raggedright \textbf{bscan}()

    \vspace{-1.5ex}

    \rule{\textwidth}{0.5\fboxrule}
    This acquires a fowler pedestal and signal and writes the normalized 
    difference into the bkg buffer (in DV this is always buffer F).

    NOTE: This command does not save the image to disk, a subsequent 
    invocation of this command will overwrite the buffer. Image will be 
    displayed in the next available DV buffer.

    \vspace{1ex}

    \end{boxedminipage}

    \label{runrun:pedscan}
    \index{runrun \textit{(module)}!runrun.pedscan \textit{(function)}}

    \vspace{0.5ex}

    \begin{boxedminipage}{\textwidth}

    \raggedright \textbf{pedscan}()

    \vspace{-1.5ex}

    \rule{\textwidth}{0.5\fboxrule}
    This acquires a pedestal frame writes it into a ped buffer.

    NOTE: This command does not save the image to disk, a subsequent 
    invocation of this command will overwrite the buffer. Image will be 
    displayed in the next available DV buffer.

    \vspace{1ex}

    \end{boxedminipage}

    \label{runrun:sigscan}
    \index{runrun \textit{(module)}!runrun.sigscan \textit{(function)}}

    \vspace{0.5ex}

    \begin{boxedminipage}{\textwidth}

    \raggedright \textbf{sigscan}()

    \vspace{-1.5ex}

    \rule{\textwidth}{0.5\fboxrule}
    This acquires a signal frame and writes it into a sig buffer.

    NOTE: This command does not save the image to disk, a subsequent 
    invocation of this command will overwrite the buffer. Image will be 
    displayed in the next available DV buffer.

    \vspace{1ex}

    \end{boxedminipage}

    \label{runrun:srun}
    \index{runrun \textit{(module)}!runrun.srun \textit{(function)}}

    \vspace{0.5ex}

    \begin{boxedminipage}{\textwidth}

    \raggedright \textbf{srun}()

    \vspace{-1.5ex}

    \rule{\textwidth}{0.5\fboxrule}
    This acquires a fowler pedestal and signal and writes the normalized 
    difference into a src buffer.

    NOTE: This command will save the image to disk using an incrementing 
    filename scheme and displayed in the next available DV buffer.

    \vspace{1ex}

    \end{boxedminipage}

    \label{runrun:brun}
    \index{runrun \textit{(module)}!runrun.brun \textit{(function)}}

    \vspace{0.5ex}

    \begin{boxedminipage}{\textwidth}

    \raggedright \textbf{brun}()

    \vspace{-1.5ex}

    \rule{\textwidth}{0.5\fboxrule}
    This acquires a fowler pedestal and signal and writes the normalized 
    difference into a bkg buffer.

    NOTE: This command will save the image to disk using an incrementing 
    filename scheme and displayed in the next available DV buffer.

    \vspace{1ex}

    \end{boxedminipage}

    \label{runrun:pedrun}
    \index{runrun \textit{(module)}!runrun.pedrun \textit{(function)}}

    \vspace{0.5ex}

    \begin{boxedminipage}{\textwidth}

    \raggedright \textbf{pedrun}()

    \vspace{-1.5ex}

    \rule{\textwidth}{0.5\fboxrule}
    This acquires a pedestal frame and writes it into a ped buffer.

    NOTE: This command will save the image to disk using an incrementing 
    filename scheme and displayed in the next available DV buffer.

    \vspace{1ex}

    \end{boxedminipage}

    \label{runrun:rrun}
    \index{runrun \textit{(module)}!runrun.rrun \textit{(function)}}

    \vspace{0.5ex}

    \begin{boxedminipage}{\textwidth}

    \raggedright \textbf{rrun}()

    \vspace{-1.5ex}

    \rule{\textwidth}{0.5\fboxrule}
    OBSOLETE: Do not use, will be removed in future versions.

    \vspace{1ex}

    \end{boxedminipage}

    \label{runrun:sigrun}
    \index{runrun \textit{(module)}!runrun.sigrun \textit{(function)}}

    \vspace{0.5ex}

    \begin{boxedminipage}{\textwidth}

    \raggedright \textbf{sigrun}()

    \vspace{-1.5ex}

    \rule{\textwidth}{0.5\fboxrule}
    This acquires a signal frame and writes into into a sig buffer.

    NOTE: This command will save the image to disk using an incrementing 
    filename scheme and displayed in the next available DV buffer.

    \vspace{1ex}

    \end{boxedminipage}

    \label{runrun:crun}
    \index{runrun \textit{(module)}!runrun.crun \textit{(function)}}

    \vspace{0.5ex}

    \begin{boxedminipage}{\textwidth}

    \raggedright \textbf{crun}(\textit{wfile}=\texttt{sys.stdout})

    \vspace{-1.5ex}

    \rule{\textwidth}{0.5\fboxrule}
    Run the array continuously. Type 'burst' to abort. This command will 
    return immediatly and run in the background. Type 'burst' to enter 
    'burst' mode which ends 'crun' mode.

    \vspace{1ex}

    \end{boxedminipage}

    \label{runrun:scandir}
    \index{runrun \textit{(module)}!runrun.scandir \textit{(function)}}

    \vspace{0.5ex}

    \begin{boxedminipage}{\textwidth}

    \raggedright \textbf{scandir}(\textit{names}=\texttt{["itime","nsamp","vreset","dsub","TEMP","COMMENT1",]}, \textit{wfile}=\texttt{sys.stdout})

    \vspace{-1.5ex}

    \rule{\textwidth}{0.5\fboxrule}
    Writes a scandir text log for the acquired images in a run. It defaults
    to cataloging the following fields: 
    "itime","nsamp","vreset","dsub","TEMP","COMMENT1".

    The list of names is a default one, and a different list could be 
    passed in.

    \vspace{1ex}

    \end{boxedminipage}

    \label{runrun:bgscan}
    \index{runrun \textit{(module)}!runrun.bgscan \textit{(function)}}

    \vspace{0.5ex}

    \begin{boxedminipage}{\textwidth}

    \raggedright \textbf{bgscan}()

    \vspace{-1.5ex}

    \rule{\textwidth}{0.5\fboxrule}
    Perform a 'sscan' in a background thread. When invoked, the command 
    will return immediately to a command prompt.

    \vspace{1ex}

    \end{boxedminipage}


%%%%%%%%%%%%%%%%%%%%%%%%%%%%%%%%%%%%%%%%%%%%%%%%%%%%%%%%%%%%%%%%%%%%%%%%%%%
%%                               Variables                               %%
%%%%%%%%%%%%%%%%%%%%%%%%%%%%%%%%%%%%%%%%%%%%%%%%%%%%%%%%%%%%%%%%%%%%%%%%%%%

  \subsection{Variables}

\begin{longtable}{|p{.30\textwidth}|p{.62\textwidth}|l}
\cline{1-2}
\cline{1-2} \centering \textbf{Name} & \centering \textbf{Description}& \\
\cline{1-2}
\endhead\cline{1-2}\multicolumn{3}{r}{\small\textit{continued on next page}}\\\endfoot\cline{1-2}
\endlastfoot\raggedright s\-c\-a\-n\- & \textbf{Value:} 
{\tt {\textless}\-f\-u\-n\-c\-t\-i\-o\-n\-~\-s\-c\-a\-n\-~\-a\-t\-~\-0\-x\-f\-6\-c\-8\-c\-8\-4\-4\-{\textgreater}\-}&\\
\cline{1-2}
\raggedright r\-u\-n\- & \textbf{Value:} 
{\tt {\textless}\-f\-u\-n\-c\-t\-i\-o\-n\-~\-r\-u\-n\-~\-a\-t\-~\-0\-x\-f\-6\-c\-8\-c\-8\-b\-4\-{\textgreater}\-}&\\
\cline{1-2}
\raggedright s\-s\-c\-a\-n\- & \textbf{Value:} 
{\tt {\textless}\-f\-u\-n\-c\-t\-i\-o\-n\-~\-s\-s\-c\-a\-n\-~\-a\-t\-~\-0\-x\-f\-6\-c\-8\-c\-9\-2\-4\-{\textgreater}\-}&\\
\cline{1-2}
\raggedright r\-s\-c\-a\-n\- & \textbf{Value:} 
{\tt {\textless}\-f\-u\-n\-c\-t\-i\-o\-n\-~\-r\-s\-c\-a\-n\-~\-a\-t\-~\-0\-x\-f\-6\-c\-8\-c\-9\-9\-4\-{\textgreater}\-}&\\
\cline{1-2}
\raggedright s\-u\-t\-r\- & \textbf{Value:} 
{\tt {\textless}\-f\-u\-n\-c\-t\-i\-o\-n\-~\-s\-u\-t\-r\-~\-a\-t\-~\-0\-x\-f\-6\-c\-8\-c\-b\-c\-4\-{\textgreater}\-}&\\
\cline{1-2}
\raggedright r\-s\-u\-t\-r\- & \textbf{Value:} 
{\tt {\textless}\-f\-u\-n\-c\-t\-i\-o\-n\-~\-r\-s\-u\-t\-r\-~\-a\-t\-~\-0\-x\-f\-6\-c\-8\-c\-c\-3\-4\-{\textgreater}\-}&\\
\cline{1-2}
\end{longtable}

    \index{runrun \textit{(module)}|)}
