%
% API Documentation for API Documentation
% Module det
%
% Generated by epydoc 3.0alpha2
% [Thu Jul 20 16:44:53 2006]
%

%%%%%%%%%%%%%%%%%%%%%%%%%%%%%%%%%%%%%%%%%%%%%%%%%%%%%%%%%%%%%%%%%%%%%%%%%%%
%%                          Module Description                           %%
%%%%%%%%%%%%%%%%%%%%%%%%%%%%%%%%%%%%%%%%%%%%%%%%%%%%%%%%%%%%%%%%%%%%%%%%%%%

    \index{det \textit{(module)}|(}
\section{Module det}

    \label{det}
det.py det.fth equivalent module..

The det module supports detector configuration, with persistence. This is 
the low level "static" configuration details. The names of the bias 
voltages and how to nominally set them (which dacs) The max number of rows 
and columns etc.

It is responsible for the detector files. it needs to know the detector 
name and the path to detector file directory

In that directory: 'detname'.map is maps signal names to dac numbers. 
'detname'.bias maps names to voltages. (Actually, it does clock rails .) 
plain old 'detname' is the file that initializes the other stuff.

the 'run' module does the same thing at a higher level det stuff does not 
normally get persisted. run stuff does... Things that are to be restored in
the next session are saved in 'lastrun.run'

The file used by the old system (det.fth) was order dependent. each line 
had a specific meaning. extra or missing lines are bad news.

With this package, the file is more tolerant, and line items are key - 
value pairs.

notes: the device driver itself might eventually need to be insmod'ed from 
here! (it is the equivalent of the data program.. the driver might be 
enhanced/overhauled over time)

What is a bias voltage, at this level? well, it is tuple.. of.. a name, a 
dacnumber, a conversion from dac counts to volts, a sanity check for 
values.. and something else?


%%%%%%%%%%%%%%%%%%%%%%%%%%%%%%%%%%%%%%%%%%%%%%%%%%%%%%%%%%%%%%%%%%%%%%%%%%%
%%                               Functions                               %%
%%%%%%%%%%%%%%%%%%%%%%%%%%%%%%%%%%%%%%%%%%%%%%%%%%%%%%%%%%%%%%%%%%%%%%%%%%%

  \subsection{Functions}

    \label{det:setclockpgm}
    \index{det \textit{(module)}!det.setclockpgm \textit{(function)}}

    \vspace{0.5ex}

    \begin{boxedminipage}{\textwidth}

    \raggedright \textbf{setclockpgm}(\textit{clockprogname}, \textit{raw\_input}=\texttt{{\textless}built-in function raw\_input{\textgreater}}, \textit{wfile}=\texttt{sys.stdout})

    \vspace{-1.5ex}

    \rule{\textwidth}{0.5\fboxrule}
    Resets the DSP and reboots it with new clock program.

    It tries 5 times after which it allows the program to continue even if 
    boot fails.

    \vspace{1ex}

    \end{boxedminipage}

    \label{det:init}
    \index{det \textit{(module)}!det.init \textit{(function)}}

    \vspace{0.5ex}

    \begin{boxedminipage}{\textwidth}

    \raggedright \textbf{init}()

    \vspace{-1.5ex}

    \rule{\textwidth}{0.5\fboxrule}
    Initialize the detector smart dictionary. Normally called only once.

    \vspace{1ex}

    \end{boxedminipage}

    \label{det:writeclockdac}
    \index{det \textit{(module)}!det.writeclockdac \textit{(function)}}

    \vspace{0.5ex}

    \begin{boxedminipage}{\textwidth}

    \raggedright \textbf{writeclockdac}(\textit{val}, *\textit{dacs})

    \vspace{-1.5ex}

    \rule{\textwidth}{0.5\fboxrule}
    Writes val (in millivolts) to clock DACs. DACs is a tuple list of clock
    DAC numbers.

    Checks that the voltage is ok. then converts the voltage into DAC 
    counts and writes that out.

    \vspace{1ex}

    \end{boxedminipage}

    \label{det:writebiasdacfunc}
    \index{det \textit{(module)}!det.writebiasdacfunc \textit{(function)}}

    \vspace{0.5ex}

    \begin{boxedminipage}{\textwidth}

    \raggedright \textbf{writebiasdacfunc}(*\textit{args}, **\textit{kwds})

    \vspace{-1.5ex}

    \rule{\textwidth}{0.5\fboxrule}
    Create (curry) a function to write the bias DAC specified with the 
    proper gain and offset.

    \vspace{1ex}

    \end{boxedminipage}

    \label{det:setbiases}
    \index{det \textit{(module)}!det.setbiases \textit{(function)}}

    \vspace{0.5ex}

    \begin{boxedminipage}{\textwidth}

    \raggedright \textbf{setbiases}(\textit{biasrunfile}, \textit{wfile}=\texttt{sys.stdout}, \textit{zero}=\texttt{False})

    \vspace{-1.5ex}

    \rule{\textwidth}{0.5\fboxrule}
    Loads biasrunfile and sets bias voltages accordingly.

    Sets up the bias rails in the order that it finds them in the file. if 
    zero==True, zero all the voltages in the reverse order of the file.

    \vspace{1ex}

    \end{boxedminipage}

    \label{det:showbiases}
    \index{det \textit{(module)}!det.showbiases \textit{(function)}}

    \vspace{0.5ex}

    \begin{boxedminipage}{\textwidth}

    \raggedright \textbf{showbiases}(\textit{wfile}=\texttt{sys.stdout})

    \vspace{-1.5ex}

    \rule{\textwidth}{0.5\fboxrule}
    List the bias names and their corresponding voltages

    \vspace{1ex}

    \end{boxedminipage}

    \label{det:loadbiasmap}
    \index{det \textit{(module)}!det.loadbiasmap \textit{(function)}}

    \vspace{0.5ex}

    \begin{boxedminipage}{\textwidth}

    \raggedright \textbf{loadbiasmap}(\textit{biasmapfile}, \textit{wfile}=\texttt{sys.stdout})

    \vspace{-1.5ex}

    \rule{\textwidth}{0.5\fboxrule}
    Load a biasmapfile, which maps bias names to dac numbers.

    each bias (programmable voltage) has: 1: a name we refer to it with. 
    (the key) 2: a tuple of dacs that it is associated with (args for the 
    set function) 3: a function that it calls to change the voltage. 4: the
    current value that it is set to.

    a pseudo-bias (which may move around several biases in a coordinated 
    manner) may be possible using this same thing..

    \vspace{1ex}

    \end{boxedminipage}

    \label{det:get_detname}
    \index{det \textit{(module)}!det.get\_detname \textit{(function)}}

    \vspace{0.5ex}

    \begin{boxedminipage}{\textwidth}

    \raggedright \textbf{get\_detname}()

    \vspace{-1.5ex}

    \rule{\textwidth}{0.5\fboxrule}
    Return the current detector name

    \vspace{1ex}

    \end{boxedminipage}

    \label{det:savedet}
    \index{det \textit{(module)}!det.savedet \textit{(function)}}

    \vspace{0.5ex}

    \begin{boxedminipage}{\textwidth}

    \raggedright \textbf{savedet}(\textit{wfile}=\texttt{sys.stdout})

    \vspace{-1.5ex}

    \rule{\textwidth}{0.5\fboxrule}
    Save the detector information in the "detfile."

    The current det dictionary's detname is used for the name of the file. 
    If a file of this name on the proper path can be opened for writing, 
    the file is written using the current entries in the detector 
    dictionary.

    \vspace{1ex}

    \end{boxedminipage}

    \label{det:loaddet}
    \index{det \textit{(module)}!det.loaddet \textit{(function)}}

    \vspace{0.5ex}

    \begin{boxedminipage}{\textwidth}

    \raggedright \textbf{loaddet}(\textit{detfilename}, \textit{wfile}=\texttt{sys.stdout})

    \vspace{-1.5ex}

    \rule{\textwidth}{0.5\fboxrule}
    Load the detector configuration from the specified detfile.

    First, it loads the biasmap. then it opens and reads the detfile 
    itself, assigning the values into the det dictionary, which is 'smart' 
    and runs code on some assignments.

    The first item in the detfile (after detname) is the clock program 
    name. assigning the clock program name into the dictionary actually 
    resets the dsp and loads the named clock program It silently ignores 
    keys in the detfile that it does not recognize.

    After reading the detfile, loaddet sets the bias voltages.

    \vspace{1ex}

    \end{boxedminipage}

    \label{det:powerup}
    \index{det \textit{(module)}!det.powerup \textit{(function)}}

    \vspace{0.5ex}

    \begin{boxedminipage}{\textwidth}

    \raggedright \textbf{powerup}(\textit{wfile}=\texttt{sys.stdout})

    \vspace{-1.5ex}

    \rule{\textwidth}{0.5\fboxrule}
    Turn on all of the biases and clocks. use the order specified in 
    detname.bias file

    \vspace{1ex}

    \end{boxedminipage}

    \label{det:powerdown}
    \index{det \textit{(module)}!det.powerdown \textit{(function)}}

    \vspace{0.5ex}

    \begin{boxedminipage}{\textwidth}

    \raggedright \textbf{powerdown}(\textit{wfile}=\texttt{sys.stdout})

    \vspace{-1.5ex}

    \rule{\textwidth}{0.5\fboxrule}
    Turn off all of the biases and clocks. use the order specified in 
    detname.bias file, in reverse.

    \vspace{1ex}

    \end{boxedminipage}


%%%%%%%%%%%%%%%%%%%%%%%%%%%%%%%%%%%%%%%%%%%%%%%%%%%%%%%%%%%%%%%%%%%%%%%%%%%
%%                               Variables                               %%
%%%%%%%%%%%%%%%%%%%%%%%%%%%%%%%%%%%%%%%%%%%%%%%%%%%%%%%%%%%%%%%%%%%%%%%%%%%

  \subsection{Variables}

\begin{longtable}{|p{.30\textwidth}|p{.62\textwidth}|l}
\cline{1-2}
\cline{1-2} \centering \textbf{Name} & \centering \textbf{Description}& \\
\cline{1-2}
\endhead\cline{1-2}\multicolumn{3}{r}{\small\textit{continued on next page}}\\\endfoot\cline{1-2}
\endlastfoot\raggedright \_\-\_\-v\-e\-r\-s\-i\-o\-n\-\_\-\_\- & \textbf{Value:} 
{\tt '\-\$\-I\-d\-:\-~\-d\-e\-t\-.\-p\-y\-~\-4\-0\-0\-~\-2\-0\-0\-6\--\-0\-6\--\-1\-9\-~\-2\-2\-:\-3\-9\-:\-3\-0\-Z\-~\-d\-r\-e\-w\-~\-\$\-~\-'\-}&\\
\cline{1-2}
\raggedright \_\-\_\-a\-u\-t\-h\-o\-r\-\_\-\_\- & \textbf{Value:} 
{\tt '\-\$\-A\-u\-t\-h\-o\-r\-:\-~\-d\-r\-e\-w\-~\-\$\-'\-}&\\
\cline{1-2}
\raggedright \_\-\_\-U\-R\-L\-\_\-\_\- & \textbf{Value:} 
{\tt '\-\$\-U\-R\-L\-:\-~\-h\-t\-t\-p\-:\-/\-/\-a\-s\-t\-r\-o\-.\-p\-a\-s\-.\-r\-o\-c\-h\-e\-s\-t\-e\-r\-.\-e\-d\-u\-/\-s\-v\-n\-/\-p\-y\-d\-s\-p\-/\-t\-r\-u\-n\-k\-/\-p\-y\-d\-s\-p\-/\-d\-e\-t\-.\-p\-y\-~\-\$\-'\-}&\\
\cline{1-2}
\raggedright d\-d\- & \textbf{Value:} 
{\tt \{\-'\-v\-b\-i\-a\-s\-'\-:\-~\-0\-,\-~\-'\-v\-r\-e\-s\-e\-t\-'\-:\-~\-0\-,\-~\-'\-m\-y\-b\-i\-a\-s\-'\-:\-~\-0\-,\-~\-'\-v\-o\-f\-f\-s\-e\-t\-'\-:\-~\-0\-\}\-}&\\
\cline{1-2}
\raggedright b\-i\-a\-s\-l\-i\-s\-t\- & \textbf{Value:} 
{\tt [\-]\-}&\\
\cline{1-2}
\raggedright d\-a\-c\-\_\-m\-v\-\_\-p\-e\-r\-\_\-c\-o\-u\-n\-t\- & \textbf{Value:} 
{\tt \{\-0\-:\-~\-1\-.\-0\-,\-~\-1\-:\-~\-1\-.\-0\-,\-~\-1\-2\-:\-~\-1\-.\-0\-,\-~\-1\-3\-:\-~\-1\-.\-0\-,\-~\-2\-6\-:\-~\-1\-.\-0\-,\-~\-2\-7\-:\-~\-1\-.\-0\-\}\-}&\\
\cline{1-2}
\raggedright d\-a\-c\-\_\-m\-v\-\_\-o\-f\-f\-s\-e\-t\- & \textbf{Value:} 
{\tt \{\-0\-:\-~\--\-1\-.\-0\-,\-~\-1\-:\-~\--\-1\-.\-0\-,\-~\-1\-2\-:\-~\-2\-.\-0\-,\-~\-1\-3\-:\-~\-2\-.\-0\-\}\-}&\\
\cline{1-2}
\raggedright m\-a\-x\-d\-a\-c\- & \textbf{Value:} 
{\tt 4\-5\-0\-0\-}&\\
\cline{1-2}
\raggedright m\-i\-n\-d\-a\-c\- & \textbf{Value:} 
{\tt -\-8\-0\-0\-0\-}&\\
\cline{1-2}
\raggedright d\-a\-c\-f\-u\-n\-c\-s\- & \textbf{Value:} 
{\tt \{\-'\-B\-I\-A\-S\-'\-:\-~\-{\textless}\-f\-u\-n\-c\-t\-i\-o\-n\-~\-w\-r\-i\-t\-e\-b\-i\-a\-s\-d\-a\-c\-f\-u\-n\-c\-~\-a\-t\-~\-0\-x\-f\-6\-e\-5\-9\-4\-c\-4\-{\textgreater}\-,\-~\-'\-C\-L\-O\-C\-K\-'\-:\-~\-{\textless}\-f\-u\-n\-c\-t\-i\-o\-n\-~\-w\-r\-i\-t\-e\-c\-.\-.\-.\-}&\\
\cline{1-2}
\raggedright d\-a\-c\-f\-u\-n\-c\-n\-a\-m\-e\-s\- & \textbf{Value:} 
{\tt [\-'\-B\-I\-A\-S\-'\-,\-~\-'\-C\-L\-O\-C\-K\-'\-]\-}&\\
\cline{1-2}
\end{longtable}


%%%%%%%%%%%%%%%%%%%%%%%%%%%%%%%%%%%%%%%%%%%%%%%%%%%%%%%%%%%%%%%%%%%%%%%%%%%
%%                           Class Description                           %%
%%%%%%%%%%%%%%%%%%%%%%%%%%%%%%%%%%%%%%%%%%%%%%%%%%%%%%%%%%%%%%%%%%%%%%%%%%%

    \index{det \textit{(module)}!det.DetDict \textit{(class)}|(}
\subsection{Class DetDict}

    \label{det:DetDict}
\begin{tabular}{cccccccc}
% Line for object, linespec=[False, False]
\multicolumn{2}{r}{\settowidth{\BCL}{object}\multirow{2}{\BCL}{object}}
&&
&&
  \\\cline{3-3}
  &&\multicolumn{1}{c|}{}
&&
&&
  \\
% Line for DataDict.DataDict, linespec=[False]
\multicolumn{4}{r}{\settowidth{\BCL}{DataDict.DataDict}\multirow{2}{\BCL}{DataDict.DataDict}}
&&
  \\\cline{5-5}
  &&&&\multicolumn{1}{c|}{}
&&
  \\
&&&&\multicolumn{2}{l}{\textbf{det.DetDict}}
\end{tabular}

Smart dictionary that contains the detector basic parameters.

Getting and setting values in this dictionary may actually read or write 
the physical hardware. Class is do-nothing, but properties can be added to 
the class on-the-fly outside of the class statement.


%%%%%%%%%%%%%%%%%%%%%%%%%%%%%%%%%%%%%%%%%%%%%%%%%%%%%%%%%%%%%%%%%%%%%%%%%%%
%%                                Methods                                %%
%%%%%%%%%%%%%%%%%%%%%%%%%%%%%%%%%%%%%%%%%%%%%%%%%%%%%%%%%%%%%%%%%%%%%%%%%%%

  \subsubsection{Methods}

    \label{DataDict:DataDict:__cmp__}
    \index{DataDict \textit{(module)}!DataDict.DataDict \textit{(class)}!DataDict.DataDict.\_\_cmp\_\_ \textit{(method)}}

    \vspace{0.5ex}

    \begin{boxedminipage}{\textwidth}

    \raggedright \textbf{\_\_cmp\_\_}(\textit{self}, \textit{dict})

    \end{boxedminipage}

    \label{DataDict:DataDict:__contains__}
    \index{DataDict \textit{(module)}!DataDict.DataDict \textit{(class)}!DataDict.DataDict.\_\_contains\_\_ \textit{(method)}}

    \vspace{0.5ex}

    \begin{boxedminipage}{\textwidth}

    \raggedright \textbf{\_\_contains\_\_}(\textit{self}, \textit{key})

    \end{boxedminipage}

    \label{object:__delattr__}
    \index{object.\_\_delattr\_\_ \textit{(function)}}

    \vspace{0.5ex}

    \begin{boxedminipage}{\textwidth}

    \raggedright \textbf{\_\_delattr\_\_}(\textit{...})

    \vspace{-1.5ex}

    \rule{\textwidth}{0.5\fboxrule}
    x.\_\_delattr\_\_('name') {\textless}=={\textgreater} del x.name

    \vspace{1ex}

    \end{boxedminipage}

    \label{DataDict:DataDict:__delitem__}
    \index{DataDict \textit{(module)}!DataDict.DataDict \textit{(class)}!DataDict.DataDict.\_\_delitem\_\_ \textit{(method)}}

    \vspace{0.5ex}

    \begin{boxedminipage}{\textwidth}

    \raggedright \textbf{\_\_delitem\_\_}(\textit{self}, \textit{key})

    \end{boxedminipage}

    \label{DataDict:DataDict:__getattr__}
    \index{DataDict \textit{(module)}!DataDict.DataDict \textit{(class)}!DataDict.DataDict.\_\_getattr\_\_ \textit{(method)}}

    \vspace{0.5ex}

    \begin{boxedminipage}{\textwidth}

    \raggedright \textbf{\_\_getattr\_\_}(\textit{self}, \textit{name})

    \end{boxedminipage}

    \label{object:__getattribute__}
    \index{object.\_\_getattribute\_\_ \textit{(function)}}

    \vspace{0.5ex}

    \begin{boxedminipage}{\textwidth}

    \raggedright \textbf{\_\_getattribute\_\_}(\textit{...})

    \vspace{-1.5ex}

    \rule{\textwidth}{0.5\fboxrule}
    x.\_\_getattribute\_\_('name') {\textless}=={\textgreater} x.name

    \vspace{1ex}

    \end{boxedminipage}

    \label{DataDict:DataDict:__getitem__}
    \index{DataDict \textit{(module)}!DataDict.DataDict \textit{(class)}!DataDict.DataDict.\_\_getitem\_\_ \textit{(method)}}

    \vspace{0.5ex}

    \begin{boxedminipage}{\textwidth}

    \raggedright \textbf{\_\_getitem\_\_}(\textit{self}, \textit{name})

    \vspace{-1.5ex}

    \rule{\textwidth}{0.5\fboxrule}
    Recall a value from the smart dictionary.

    If the name has a getfunc, that function is called (with arguments if 
    arguments are defined for the name; the arguments are the same for 
    setfunc and getfunc) The return value of getfunc is returned to the 
    user. If no getfunc exists, the internal value associated with that 
    name is returned.

    \vspace{1ex}

    \end{boxedminipage}

    \label{object:__hash__}
    \index{object.\_\_hash\_\_ \textit{(function)}}

    \vspace{0.5ex}

    \begin{boxedminipage}{\textwidth}

    \raggedright \textbf{\_\_hash\_\_}(\textit{x})

    \vspace{-1.5ex}

    \rule{\textwidth}{0.5\fboxrule}
    hash(x)

    \vspace{1ex}

    \end{boxedminipage}

    \vspace{0.5ex}

    \begin{boxedminipage}{\textwidth}

    \raggedright \textbf{\_\_init\_\_}(\textit{self}, \textit{dict}=\texttt{None})

    x.\_\_init\_\_(...) initializes x; see x.\_\_class\_\_.\_\_doc\_\_ for 
    signature

    \vspace{1ex}

      Overrides: object.\_\_init\_\_ 	extit{(inherited documentation)}

    \end{boxedminipage}

    \label{DataDict:DataDict:__iter__}
    \index{DataDict \textit{(module)}!DataDict.DataDict \textit{(class)}!DataDict.DataDict.\_\_iter\_\_ \textit{(method)}}

    \vspace{0.5ex}

    \begin{boxedminipage}{\textwidth}

    \raggedright \textbf{\_\_iter\_\_}(\textit{self})

    \end{boxedminipage}

    \label{DataDict:DataDict:__len__}
    \index{DataDict \textit{(module)}!DataDict.DataDict \textit{(class)}!DataDict.DataDict.\_\_len\_\_ \textit{(method)}}

    \vspace{0.5ex}

    \begin{boxedminipage}{\textwidth}

    \raggedright \textbf{\_\_len\_\_}(\textit{self})

    \end{boxedminipage}

    \label{object:__new__}
    \index{object.\_\_new\_\_ \textit{(function)}}

    \vspace{0.5ex}

    \begin{boxedminipage}{\textwidth}

    \raggedright \textbf{\_\_new\_\_}(\textit{T}, \textit{S}, \textit{...})

    \vspace{1ex}

      \textbf{Return Value}
      \begin{quote}
\begin{alltt}
a new object with type S, a subtype of T
\end{alltt}

      \end{quote}

    \vspace{1ex}

    \end{boxedminipage}

    \label{object:__reduce__}
    \index{object.\_\_reduce\_\_ \textit{(function)}}

    \vspace{0.5ex}

    \begin{boxedminipage}{\textwidth}

    \raggedright \textbf{\_\_reduce\_\_}(\textit{...})

    \vspace{-1.5ex}

    \rule{\textwidth}{0.5\fboxrule}
    helper for pickle

    \vspace{1ex}

    \end{boxedminipage}

    \label{object:__reduce_ex__}
    \index{object.\_\_reduce\_ex\_\_ \textit{(function)}}

    \vspace{0.5ex}

    \begin{boxedminipage}{\textwidth}

    \raggedright \textbf{\_\_reduce\_ex\_\_}(\textit{...})

    \vspace{-1.5ex}

    \rule{\textwidth}{0.5\fboxrule}
    helper for pickle

    \vspace{1ex}

    \end{boxedminipage}

    \vspace{0.5ex}

    \begin{boxedminipage}{\textwidth}

    \raggedright \textbf{\_\_repr\_\_}(\textit{self})

    repr(x)

    \vspace{1ex}

      Overrides: object.\_\_repr\_\_ 	extit{(inherited documentation)}

    \end{boxedminipage}

    \vspace{0.5ex}

    \begin{boxedminipage}{\textwidth}

    \raggedright \textbf{\_\_setattr\_\_}(\textit{self}, \textit{name}, \textit{value})

    x.\_\_setattr\_\_('name', value) {\textless}=={\textgreater} x.name = 
    value

    \vspace{1ex}

      Overrides: object.\_\_setattr\_\_ 	extit{(inherited documentation)}

    \end{boxedminipage}

    \label{DataDict:DataDict:__setitem__}
    \index{DataDict \textit{(module)}!DataDict.DataDict \textit{(class)}!DataDict.DataDict.\_\_setitem\_\_ \textit{(method)}}

    \vspace{0.5ex}

    \begin{boxedminipage}{\textwidth}

    \raggedright \textbf{\_\_setitem\_\_}(\textit{self}, \textit{name}, \textit{val})

    \vspace{-1.5ex}

    \rule{\textwidth}{0.5\fboxrule}
    Assign a value to the smart dictionary.

    If a setfunc exists, that is called first (with args if they exist). 
    Successful setfunc execution is followed by remembering the value that 
    was written, then by an attempt to update a GUI via set\_widget (the 
    default set\_widget function does nothing.) The namemap can map 
    dictionary names to widget names if they are different; exceptions in 
    namemap or set\_widget are ignored.

    \vspace{1ex}

    \end{boxedminipage}

    \label{object:__str__}
    \index{object.\_\_str\_\_ \textit{(function)}}

    \vspace{0.5ex}

    \begin{boxedminipage}{\textwidth}

    \raggedright \textbf{\_\_str\_\_}(\textit{x})

    \vspace{-1.5ex}

    \rule{\textwidth}{0.5\fboxrule}
    str(x)

    \vspace{1ex}

    \end{boxedminipage}

    \label{DataDict:DataDict:additem}
    \index{DataDict \textit{(module)}!DataDict.DataDict \textit{(class)}!DataDict.DataDict.additem \textit{(method)}}

    \vspace{0.5ex}

    \begin{boxedminipage}{\textwidth}

    \raggedright \textbf{additem}(\textit{self}, \textit{name}, \textit{val}=\texttt{None}, \textit{setfunc}=\texttt{None}, \textit{getfunc}=\texttt{None}, \textit{args}=\texttt{()}, \textit{kwds}=\texttt{None}, \textit{docstring}=\texttt{None})

    \vspace{-1.5ex}

    \rule{\textwidth}{0.5\fboxrule}
    Put a new entry in the smart dictionary.

    name = string used to identify this thing val = initial value of item 
    setfunc: callable. If defined, it is called with new value on writes 
    getfunc: callable. If defined, it is called on reads args: if defined, 
    passed as last argument for setfunc and getfunc. docstring: help string
    for the item.

    \vspace{1ex}

    \end{boxedminipage}

    \label{DataDict:DataDict:clear}
    \index{DataDict \textit{(module)}!DataDict.DataDict \textit{(class)}!DataDict.DataDict.clear \textit{(method)}}

    \vspace{0.5ex}

    \begin{boxedminipage}{\textwidth}

    \raggedright \textbf{clear}(\textit{self})

    \end{boxedminipage}

    \label{DataDict:DataDict:copy}
    \index{DataDict \textit{(module)}!DataDict.DataDict \textit{(class)}!DataDict.DataDict.copy \textit{(method)}}

    \vspace{0.5ex}

    \begin{boxedminipage}{\textwidth}

    \raggedright \textbf{copy}(\textit{self})

    \end{boxedminipage}

    \label{DataDict:DataDict:get}
    \index{DataDict \textit{(module)}!DataDict.DataDict \textit{(class)}!DataDict.DataDict.get \textit{(method)}}

    \vspace{0.5ex}

    \begin{boxedminipage}{\textwidth}

    \raggedright \textbf{get}(\textit{self}, \textit{key}, \textit{failobj}=\texttt{None})

    \end{boxedminipage}

    \label{DataDict:DataDict:has_key}
    \index{DataDict \textit{(module)}!DataDict.DataDict \textit{(class)}!DataDict.DataDict.has\_key \textit{(method)}}

    \vspace{0.5ex}

    \begin{boxedminipage}{\textwidth}

    \raggedright \textbf{has\_key}(\textit{self}, \textit{key})

    \end{boxedminipage}

    \label{DataDict:DataDict:items}
    \index{DataDict \textit{(module)}!DataDict.DataDict \textit{(class)}!DataDict.DataDict.items \textit{(method)}}

    \vspace{0.5ex}

    \begin{boxedminipage}{\textwidth}

    \raggedright \textbf{items}(\textit{self})

    \end{boxedminipage}

    \label{DataDict:DataDict:iteritems}
    \index{DataDict \textit{(module)}!DataDict.DataDict \textit{(class)}!DataDict.DataDict.iteritems \textit{(method)}}

    \vspace{0.5ex}

    \begin{boxedminipage}{\textwidth}

    \raggedright \textbf{iteritems}(\textit{self})

    \end{boxedminipage}

    \label{DataDict:DataDict:iterkeys}
    \index{DataDict \textit{(module)}!DataDict.DataDict \textit{(class)}!DataDict.DataDict.iterkeys \textit{(method)}}

    \vspace{0.5ex}

    \begin{boxedminipage}{\textwidth}

    \raggedright \textbf{iterkeys}(\textit{self})

    \end{boxedminipage}

    \label{DataDict:DataDict:itervalues}
    \index{DataDict \textit{(module)}!DataDict.DataDict \textit{(class)}!DataDict.DataDict.itervalues \textit{(method)}}

    \vspace{0.5ex}

    \begin{boxedminipage}{\textwidth}

    \raggedright \textbf{itervalues}(\textit{self})

    \end{boxedminipage}

    \label{DataDict:DataDict:keys}
    \index{DataDict \textit{(module)}!DataDict.DataDict \textit{(class)}!DataDict.DataDict.keys \textit{(method)}}

    \vspace{0.5ex}

    \begin{boxedminipage}{\textwidth}

    \raggedright \textbf{keys}(\textit{self})

    \end{boxedminipage}

    \label{DataDict:DataDict:popitem}
    \index{DataDict \textit{(module)}!DataDict.DataDict \textit{(class)}!DataDict.DataDict.popitem \textit{(method)}}

    \vspace{0.5ex}

    \begin{boxedminipage}{\textwidth}

    \raggedright \textbf{popitem}(\textit{self})

    \end{boxedminipage}

    \label{DataDict:DataDict:set_widget}
    \index{DataDict \textit{(module)}!DataDict.DataDict \textit{(class)}!DataDict.DataDict.set\_widget \textit{(method)}}

    \vspace{0.5ex}

    \begin{boxedminipage}{\textwidth}

    \raggedright \textbf{set\_widget}(\textit{self}, *\textit{args})

    \end{boxedminipage}

    \label{DataDict:DataDict:setdefault}
    \index{DataDict \textit{(module)}!DataDict.DataDict \textit{(class)}!DataDict.DataDict.setdefault \textit{(method)}}

    \vspace{0.5ex}

    \begin{boxedminipage}{\textwidth}

    \raggedright \textbf{setdefault}(\textit{self}, \textit{key}, \textit{failobj}=\texttt{None})

    \end{boxedminipage}

    \label{DataDict:DataDict:update}
    \index{DataDict \textit{(module)}!DataDict.DataDict \textit{(class)}!DataDict.DataDict.update \textit{(method)}}

    \vspace{0.5ex}

    \begin{boxedminipage}{\textwidth}

    \raggedright \textbf{update}(\textit{self}, \textit{dict})

    \end{boxedminipage}

    \label{DataDict:DataDict:values}
    \index{DataDict \textit{(module)}!DataDict.DataDict \textit{(class)}!DataDict.DataDict.values \textit{(method)}}

    \vspace{0.5ex}

    \begin{boxedminipage}{\textwidth}

    \raggedright \textbf{values}(\textit{self})

    \end{boxedminipage}


%%%%%%%%%%%%%%%%%%%%%%%%%%%%%%%%%%%%%%%%%%%%%%%%%%%%%%%%%%%%%%%%%%%%%%%%%%%
%%                            Class Variables                            %%
%%%%%%%%%%%%%%%%%%%%%%%%%%%%%%%%%%%%%%%%%%%%%%%%%%%%%%%%%%%%%%%%%%%%%%%%%%%

  \subsubsection{Class Variables}

\begin{longtable}{|p{.30\textwidth}|p{.62\textwidth}|l}
\cline{1-2}
\cline{1-2} \centering \textbf{Name} & \centering \textbf{Description}& \\
\cline{1-2}
\endhead\cline{1-2}\multicolumn{3}{r}{\small\textit{continued on next page}}\\\endfoot\cline{1-2}
\endlastfoot\raggedright \_\-\_\-c\-l\-a\-s\-s\-\_\-\_\- & \textbf{Value:} 
{\tt {\textless}\-a\-t\-t\-r\-i\-b\-u\-t\-e\-~\-'\-\_\-\_\-c\-l\-a\-s\-s\-\_\-\_\-'\-~\-o\-f\-~\-'\-o\-b\-j\-e\-c\-t\-'\-~\-o\-b\-j\-e\-c\-t\-s\-{\textgreater}\-}&\\
\cline{1-2}
\raggedright n\-a\-m\-e\-m\-a\-p\- & \textbf{Value:} 
{\tt \{\-\}\-}&\\
\cline{1-2}
\end{longtable}

    \index{det \textit{(module)}!det.DetDict \textit{(class)}|)}
    \index{det \textit{(module)}|)}
