%
% API Documentation for API Documentation
% Module loopback
%
% Generated by epydoc 3.0alpha2
% [Tue Jul 18 17:01:02 2006]
%

%%%%%%%%%%%%%%%%%%%%%%%%%%%%%%%%%%%%%%%%%%%%%%%%%%%%%%%%%%%%%%%%%%%%%%%%%%%
%%                          Module Description                           %%
%%%%%%%%%%%%%%%%%%%%%%%%%%%%%%%%%%%%%%%%%%%%%%%%%%%%%%%%%%%%%%%%%%%%%%%%%%%

    \index{loopback \textit{(module)}|(}
\section{Module loopback}

    \label{loopback}
Two loopback tests for testing the PCI card.

loopback() just uses write and read. irqloopback uses the Half-Full flag 
interrupt and the device driver procsssing bootloop actually boots the dsp.

usage: cd to pydsp source directory, typically 'cd dsp/pydsp'

\# python -i loopback.py {\textgreater}{\textgreater}{\textgreater} 
bootloop()


%%%%%%%%%%%%%%%%%%%%%%%%%%%%%%%%%%%%%%%%%%%%%%%%%%%%%%%%%%%%%%%%%%%%%%%%%%%
%%                               Functions                               %%
%%%%%%%%%%%%%%%%%%%%%%%%%%%%%%%%%%%%%%%%%%%%%%%%%%%%%%%%%%%%%%%%%%%%%%%%%%%

  \subsection{Functions}

    \label{loopback:loopone}
    \index{loopback \textit{(module)}!loopback.loopone \textit{(function)}}

    \vspace{0.5ex}

    \begin{boxedminipage}{\textwidth}

    \raggedright \textbf{loopone}(\textit{i}=\texttt{0}, \textit{n}=\texttt{100})

    \vspace{-1.5ex}

    \rule{\textwidth}{0.5\fboxrule}
    simple loopback test. write i to output, read iback from fifo directly.
    do this n times. return nuber of errors.

    \vspace{1ex}

    \end{boxedminipage}

    \label{loopback:loopback}
    \index{loopback \textit{(module)}!loopback.loopback \textit{(function)}}

    \vspace{0.5ex}

    \begin{boxedminipage}{\textwidth}

    \raggedright \textbf{loopback}()

    \vspace{-1.5ex}

    \rule{\textwidth}{0.5\fboxrule}
    simple loopback test. write to output, read from fifo directly.

    \vspace{1ex}

    \end{boxedminipage}

    \label{loopback:irqloopback}
    \index{loopback \textit{(module)}!loopback.irqloopback \textit{(function)}}

    \vspace{0.5ex}

    \begin{boxedminipage}{\textwidth}

    \raggedright \textbf{irqloopback}(\textit{nrow}=\texttt{256}, \textit{ncol}=\texttt{256}, \textit{nsamp}=\texttt{2})

    \vspace{-1.5ex}

    \rule{\textwidth}{0.5\fboxrule}
    set image parameters. Tell driver to expect video. then write the video
    directly out the HSS. loopback plug sends it back into fifo, and the 
    driver reads it. go to the raw image buffers to test that video is 
    actually coming back. A "loopbackdsp" object could be of use here, and 
    simulate the imager.

    \vspace{1ex}

    \end{boxedminipage}

    \label{loopback:progloopback}
    \index{loopback \textit{(module)}!loopback.progloopback \textit{(function)}}

    \vspace{0.5ex}

    \begin{boxedminipage}{\textwidth}

    \raggedright \textbf{progloopback}()

    \vspace{-1.5ex}

    \rule{\textwidth}{0.5\fboxrule}
    Test the loopback mechanism by communicating with a running dsp 
    program.

    write down all values to a DSP memory location, then read them back. 
    runs the loop approximately every 4 -10 seconds.

    \vspace{1ex}

    \end{boxedminipage}

    \label{loopback:bootloop}
    \index{loopback \textit{(module)}!loopback.bootloop \textit{(function)}}

    \vspace{0.5ex}

    \begin{boxedminipage}{\textwidth}

    \raggedright \textbf{bootloop}()

    \vspace{-1.5ex}

    \rule{\textwidth}{0.5\fboxrule}
    Continually reboot the dsp and run an echo test.

    Alternates between echo and necho. necho echoes negated values. each 
    echo test writes all values from 0 to 65535 out to the echoing program.
    The values are then read back one at a time and compared to expected. 
    If they dont match, call it a data error and move to the next reboot. 
    secs per transfer is per roundtrip, and includes the writes as well as 
    the reads.

    \vspace{1ex}

    \end{boxedminipage}


%%%%%%%%%%%%%%%%%%%%%%%%%%%%%%%%%%%%%%%%%%%%%%%%%%%%%%%%%%%%%%%%%%%%%%%%%%%
%%                               Variables                               %%
%%%%%%%%%%%%%%%%%%%%%%%%%%%%%%%%%%%%%%%%%%%%%%%%%%%%%%%%%%%%%%%%%%%%%%%%%%%

  \subsection{Variables}

\begin{longtable}{|p{.30\textwidth}|p{.62\textwidth}|l}
\cline{1-2}
\cline{1-2} \centering \textbf{Name} & \centering \textbf{Description}& \\
\cline{1-2}
\endhead\cline{1-2}\multicolumn{3}{r}{\small\textit{continued on next page}}\\\endfoot\cline{1-2}
\endlastfoot\raggedright k\- & \textbf{Value:} 
{\tt 0\-}&\\
\cline{1-2}
\end{longtable}

    \index{loopback \textit{(module)}|)}
