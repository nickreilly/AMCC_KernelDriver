%
% API Documentation for API Documentation
% Module pydsp
%
% Generated by epydoc 3.0alpha2
% [Thu Jul 20 16:44:54 2006]
%

%%%%%%%%%%%%%%%%%%%%%%%%%%%%%%%%%%%%%%%%%%%%%%%%%%%%%%%%%%%%%%%%%%%%%%%%%%%
%%                          Module Description                           %%
%%%%%%%%%%%%%%%%%%%%%%%%%%%%%%%%%%%%%%%%%%%%%%%%%%%%%%%%%%%%%%%%%%%%%%%%%%%

    \index{pydsp \textit{(module)}|(}
\section{Module pydsp}

    \label{pydsp}
pydsp.py -- the almost top level file.

Usage was originally: python -i pydsp.py Now, start\_pydsp.py is the top 
level. Usage is effectively: python -i start\_pydsp.py

Importing this file again when it was the top level caused undesired 
execution of stuff (first import.) Plus, it seemed an architectural error 
to re-import the top file.

Alias startup to 'pydsp' by inserting the line: alias pydsp='python -tt -i 
\$DSPHOME/pydsp/start\_pydsp.py' in {\textasciitilde}/.bashrc with this 
alias, just type 'pydsp' at the command line.

it emulates dspsys.fth in the dspsys forth-based system

TODO: It has a mishmash of things and could use a refactor. Its primary 
focus should be defining the namespace that the user sees. Sections of that
namespace might be sensibly broken out to separate files. The help 
functions need access to that namespace, so they wound up here.


%%%%%%%%%%%%%%%%%%%%%%%%%%%%%%%%%%%%%%%%%%%%%%%%%%%%%%%%%%%%%%%%%%%%%%%%%%%
%%                               Functions                               %%
%%%%%%%%%%%%%%%%%%%%%%%%%%%%%%%%%%%%%%%%%%%%%%%%%%%%%%%%%%%%%%%%%%%%%%%%%%%

  \subsection{Functions}

    \label{pydsp:help}
    \index{pydsp \textit{(module)}!pydsp.help \textit{(function)}}

    \vspace{0.5ex}

    \begin{boxedminipage}{\textwidth}

    \raggedright \textbf{help}(\textit{thing}=\texttt{None}, \textit{raw\_input}=\texttt{{\textless}built-in function raw\_input{\textgreater}}, \textit{wfile}=\texttt{sys.stdout})

    \vspace{-1.5ex}

    \rule{\textwidth}{0.5\fboxrule}
    Wrapper for python's pydoc help function.

    It checks input; if it is a string, check all the likely places for a 
    match: the run dict, the det dict, and the globals. If it is not a 
    string, just pass it to pydoc.

    \vspace{1ex}

    \end{boxedminipage}

    \label{pydsp:see}
    \index{pydsp \textit{(module)}!pydsp.see \textit{(function)}}

    \vspace{0.5ex}

    \begin{boxedminipage}{\textwidth}

    \raggedright \textbf{see}(\textit{thing}, \textit{wfile}=\texttt{sys.stdout})

    \vspace{-1.5ex}

    \rule{\textwidth}{0.5\fboxrule}
    See the source code associated with an item, if possible.

    It accepts strings or python objects. 'see' checks a string argument 
    against the run and det dictionaries, then against the global 
    namespace. If a match is found, it converts to a python object if 
    possible and subsequently tries to get the source code for python 
    objects.

    \vspace{1ex}

    \end{boxedminipage}

    \label{pydsp:words}
    \index{pydsp \textit{(module)}!pydsp.words \textit{(function)}}

    \vspace{0.5ex}

    \begin{boxedminipage}{\textwidth}

    \raggedright \textbf{words}(\textit{filt}=\texttt{None}, \textit{wfile}=\texttt{sys.stdout})

    \vspace{-1.5ex}

    \rule{\textwidth}{0.5\fboxrule}
    Prints three columns of words: funcs, run dict, det dict

    \vspace{1ex}

    \end{boxedminipage}

    \label{pydsp:rdclockcurrent}
    \index{pydsp \textit{(module)}!pydsp.rdclockcurrent \textit{(function)}}

    \vspace{0.5ex}

    \begin{boxedminipage}{\textwidth}

    \raggedright \textbf{rdclockcurrent}(\textit{wfile}=\texttt{sys.stdout})

    \vspace{-1.5ex}

    \rule{\textwidth}{0.5\fboxrule}
    Internal function, do not use.

    \vspace{1ex}

    \end{boxedminipage}

    \label{pydsp:rdclockrail}
    \index{pydsp \textit{(module)}!pydsp.rdclockrail \textit{(function)}}

    \vspace{0.5ex}

    \begin{boxedminipage}{\textwidth}

    \raggedright \textbf{rdclockrail}(\textit{wfile}=\texttt{sys.stdout})

    \vspace{-1.5ex}

    \rule{\textwidth}{0.5\fboxrule}
    Internal function, do not use.

    \vspace{1ex}

    \end{boxedminipage}

    \label{pydsp:rdbiascurrent}
    \index{pydsp \textit{(module)}!pydsp.rdbiascurrent \textit{(function)}}

    \vspace{0.5ex}

    \begin{boxedminipage}{\textwidth}

    \raggedright \textbf{rdbiascurrent}(\textit{wfile}=\texttt{sys.stdout})

    \vspace{-1.5ex}

    \rule{\textwidth}{0.5\fboxrule}
    Internal function, do not use.

    \vspace{1ex}

    \end{boxedminipage}

    \label{pydsp:rdbiasvoltage}
    \index{pydsp \textit{(module)}!pydsp.rdbiasvoltage \textit{(function)}}

    \vspace{0.5ex}

    \begin{boxedminipage}{\textwidth}

    \raggedright \textbf{rdbiasvoltage}(\textit{wfile}=\texttt{sys.stdout})

    \vspace{-1.5ex}

    \rule{\textwidth}{0.5\fboxrule}
    Internal function, do not use.

    \vspace{1ex}

    \end{boxedminipage}

    \label{pydsp:rbv}
    \index{pydsp \textit{(module)}!pydsp.rbv \textit{(function)}}

    \vspace{0.5ex}

    \begin{boxedminipage}{\textwidth}

    \raggedright \textbf{rbv}(\textit{wfile}=\texttt{sys.stdout})

    \end{boxedminipage}

    \label{pydsp:loaduser}
    \index{pydsp \textit{(module)}!pydsp.loaduser \textit{(function)}}

    \vspace{0.5ex}

    \begin{boxedminipage}{\textwidth}

    \raggedright \textbf{loaduser}(\textit{name}=\texttt{""}, \textit{wfile}=\texttt{sys.stdout})

    \vspace{-1.5ex}

    \rule{\textwidth}{0.5\fboxrule}
    Not implemented. To load in a user program use execuser instead.

    \vspace{1ex}

    \end{boxedminipage}

    \label{pydsp:userload}
    \index{pydsp \textit{(module)}!pydsp.userload \textit{(function)}}

    \vspace{0.5ex}

    \begin{boxedminipage}{\textwidth}

    \raggedright \textbf{userload}(\textit{name}=\texttt{""}, \textit{wfile}=\texttt{sys.stdout})

    \vspace{-1.5ex}

    \rule{\textwidth}{0.5\fboxrule}
    Not implemented. To load in a user program use execuser instead.

    \vspace{1ex}

    \end{boxedminipage}

    \label{pydsp:execuser}
    \index{pydsp \textit{(module)}!pydsp.execuser \textit{(function)}}

    \vspace{0.5ex}

    \begin{boxedminipage}{\textwidth}

    \raggedright \textbf{execuser}(\textit{filename}=\texttt{None}, \textit{raw\_input}=\texttt{{\textless}built-in function raw\_input{\textgreater}}, \textit{wfile}=\texttt{sys.stdout})

    \vspace{-1.5ex}

    \rule{\textwidth}{0.5\fboxrule}
    Run a user program from the acq directory.

    User program can be a straight python script, or file that defines a 
    bunch of new words.

    Any new words are added to the globals dictionary.

    \vspace{1ex}

    \end{boxedminipage}

    \label{pydsp:runuser}
    \index{pydsp \textit{(module)}!pydsp.runuser \textit{(function)}}

    \vspace{0.5ex}

    \begin{boxedminipage}{\textwidth}

    \raggedright \textbf{runuser}(\textit{wfile}=\texttt{sys.stdout})

    \vspace{-1.5ex}

    \rule{\textwidth}{0.5\fboxrule}
    Not implemented. To load in a user program use execuser instead.

    \vspace{1ex}

    \end{boxedminipage}

    \label{pydsp:tcstart}
    \index{pydsp \textit{(module)}!pydsp.tcstart \textit{(function)}}

    \vspace{0.5ex}

    \begin{boxedminipage}{\textwidth}

    \raggedright \textbf{tcstart}()

    \vspace{-1.5ex}

    \rule{\textwidth}{0.5\fboxrule}
    Activate temp control. Use 'tcgoto' to set the desired temp.

    NOTE: Hardware temperature control not yet implemented.

    \vspace{1ex}

    \end{boxedminipage}

    \label{pydsp:tcstop}
    \index{pydsp \textit{(module)}!pydsp.tcstop \textit{(function)}}

    \vspace{0.5ex}

    \begin{boxedminipage}{\textwidth}

    \raggedright \textbf{tcstop}()

    \vspace{-1.5ex}

    \rule{\textwidth}{0.5\fboxrule}
    Stop temp control.

    NOTE: Hardware temperature control not yet implemented.

    \vspace{1ex}

    \end{boxedminipage}

    \label{pydsp:tcgoto}
    \index{pydsp \textit{(module)}!pydsp.tcgoto \textit{(function)}}

    \vspace{0.5ex}

    \begin{boxedminipage}{\textwidth}

    \raggedright \textbf{tcgoto}(\textit{temp})

    \vspace{-1.5ex}

    \rule{\textwidth}{0.5\fboxrule}
    Set a new goal temperature.

    NOTE: Hardware temperature control not yet implemented.

    \vspace{1ex}

    \end{boxedminipage}

    \label{pydsp:tcgoal}
    \index{pydsp \textit{(module)}!pydsp.tcgoal \textit{(function)}}

    \vspace{0.5ex}

    \begin{boxedminipage}{\textwidth}

    \raggedright \textbf{tcgoal}()

    \vspace{-1.5ex}

    \rule{\textwidth}{0.5\fboxrule}
    Print the current temperature goal. (Final temp)

    NOTE: Hardware temperature control not yet implemented.

    \vspace{1ex}

    \end{boxedminipage}

    \label{pydsp:tccarrot}
    \index{pydsp \textit{(module)}!pydsp.tccarrot \textit{(function)}}

    \vspace{0.5ex}

    \begin{boxedminipage}{\textwidth}

    \raggedright \textbf{tccarrot}()

    \vspace{-1.5ex}

    \rule{\textwidth}{0.5\fboxrule}
    Print the current instantaneous temperature carrot.

    NOTE: Hardware temperature control not yet implemented.

    \vspace{1ex}

    \end{boxedminipage}

    \label{pydsp:tcwait}
    \index{pydsp \textit{(module)}!pydsp.tcwait \textit{(function)}}

    \vspace{0.5ex}

    \begin{boxedminipage}{\textwidth}

    \raggedright \textbf{tcwait}()

    \vspace{-1.5ex}

    \rule{\textwidth}{0.5\fboxrule}
    Wait for the desired temperature (carrot) to reach the goal 
    temperature.

    NOTE: Hardware temperature control not yet implemented.

    \vspace{1ex}

    \end{boxedminipage}

    \label{pydsp:temp2file}
    \index{pydsp \textit{(module)}!pydsp.temp2file \textit{(function)}}

    \vspace{0.5ex}

    \begin{boxedminipage}{\textwidth}

    \raggedright \textbf{temp2file}(\textit{filename}=\texttt{"/scratch/data/tempmon"})

    \end{boxedminipage}

    \label{pydsp:startTempControl}
    \index{pydsp \textit{(module)}!pydsp.startTempControl \textit{(function)}}

    \vspace{0.5ex}

    \begin{boxedminipage}{\textwidth}

    \raggedright \textbf{startTempControl}()

    \vspace{-1.5ex}

    \rule{\textwidth}{0.5\fboxrule}
    connect the automatic temperature control up to the system.

    \vspace{1ex}

    \end{boxedminipage}

    \label{pydsp:try_int}
    \index{pydsp \textit{(module)}!pydsp.try\_int \textit{(function)}}

    \vspace{0.5ex}

    \begin{boxedminipage}{\textwidth}

    \raggedright \textbf{try\_int}(\textit{token})

    \vspace{-1.5ex}

    \rule{\textwidth}{0.5\fboxrule}
    Try to make the token into an integer

    If it can be made an int, return the int. else, return the original 
    token.

    \vspace{1ex}

    \end{boxedminipage}

    \label{pydsp:execcmd}
    \index{pydsp \textit{(module)}!pydsp.execcmd \textit{(function)}}

    \vspace{0.5ex}

    \begin{boxedminipage}{\textwidth}

    \raggedright \textbf{execcmd}(\textit{s}, \textit{raw\_input}=\texttt{{\textless}built-in function raw\_input{\textgreater}}, \textit{wfile}=\texttt{sys.stdout})

    \vspace{-1.5ex}

    \rule{\textwidth}{0.5\fboxrule}
    Process s, a text command string.

    It might be a smart dictionary (run or det) word, or it might be a 
    function.

    Check out ipython from scipy before rewriting this.

    \vspace{1ex}

    \end{boxedminipage}

    \label{pydsp:pyshowall_connect}
    \index{pydsp \textit{(module)}!pydsp.pyshowall\_connect \textit{(function)}}

    \vspace{0.5ex}

    \begin{boxedminipage}{\textwidth}

    \raggedright \textbf{pyshowall\_connect}()

    \vspace{-1.5ex}

    \rule{\textwidth}{0.5\fboxrule}
    Connect pyshowall to the system.

    Pyshowall attempts to be generic. It is the standard interface for 
    dealing with camera arrays. It does not know about the electronics. The
    main app may have different ways of hooking into the pyshowall funcs, 
    so in this function, we plug in the application specific hooks. The 
    main app conforms to it more than it conforms to the main app.

    \vspace{1ex}

    \end{boxedminipage}

    \label{pydsp:cloop}
    \index{pydsp \textit{(module)}!pydsp.cloop \textit{(function)}}

    \vspace{0.5ex}

    \begin{boxedminipage}{\textwidth}

    \raggedright \textbf{cloop}(\textit{raw\_input}=\texttt{{\textless}built-in function raw\_input{\textgreater}}, \textit{wfile}=\texttt{sys.stdout})

    \vspace{-1.5ex}

    \rule{\textwidth}{0.5\fboxrule}
    The main pydsp command loop.

    Very simple. Replace this with ipython someday. It gets and sets things
    in the run dict or the data dict. or, runs one of the commands in this 
    file.

    \vspace{1ex}

    \end{boxedminipage}

    \label{pydsp:doCompleter}
    \index{pydsp \textit{(module)}!pydsp.doCompleter \textit{(function)}}

    \vspace{0.5ex}

    \begin{boxedminipage}{\textwidth}

    \raggedright \textbf{doCompleter}()

    \vspace{-1.5ex}

    \rule{\textwidth}{0.5\fboxrule}
    Handles tab completion.

    Hook up the readline module so that all commands are logged to a 
    history file and are available after the system shuts down and gets 
    restarted.

    \vspace{1ex}

    \end{boxedminipage}

    \label{pydsp:startWebServer}
    \index{pydsp \textit{(module)}!pydsp.startWebServer \textit{(function)}}

    \vspace{0.5ex}

    \begin{boxedminipage}{\textwidth}

    \raggedright \textbf{startWebServer}()

    \vspace{-1.5ex}

    \rule{\textwidth}{0.5\fboxrule}
    Start up a simple embedded web server in its own thread.

    \vspace{1ex}

    \end{boxedminipage}

    \label{pydsp:startpydsp}
    \index{pydsp \textit{(module)}!pydsp.startpydsp \textit{(function)}}

    \vspace{0.5ex}

    \begin{boxedminipage}{\textwidth}

    \raggedright \textbf{startpydsp}()

    \vspace{-1.5ex}

    \rule{\textwidth}{0.5\fboxrule}
    Iinitialize the system from the base defaults.

    Then recover the data that is in lastrun.run then hook up a control-c 
    interrupt handler, and then the filter wheel.

    \vspace{1ex}

    \end{boxedminipage}


%%%%%%%%%%%%%%%%%%%%%%%%%%%%%%%%%%%%%%%%%%%%%%%%%%%%%%%%%%%%%%%%%%%%%%%%%%%
%%                               Variables                               %%
%%%%%%%%%%%%%%%%%%%%%%%%%%%%%%%%%%%%%%%%%%%%%%%%%%%%%%%%%%%%%%%%%%%%%%%%%%%

  \subsection{Variables}

\begin{longtable}{|p{.30\textwidth}|p{.62\textwidth}|l}
\cline{1-2}
\cline{1-2} \centering \textbf{Name} & \centering \textbf{Description}& \\
\cline{1-2}
\endhead\cline{1-2}\multicolumn{3}{r}{\small\textit{continued on next page}}\\\endfoot\cline{1-2}
\endlastfoot\raggedright \_\-\_\-r\-e\-v\-i\-s\-i\-o\-n\-\_\-\_\- & \textbf{Value:} 
{\tt '\-\$\-I\-d\-:\-~\-p\-y\-d\-s\-p\-.\-p\-y\-~\-4\-0\-1\-~\-2\-0\-0\-6\--\-0\-7\--\-1\-1\-~\-2\-2\-:\-3\-1\-:\-5\-8\-Z\-~\-d\-r\-e\-w\-~\-\$\-'\-}&\\
\cline{1-2}
\raggedright \_\-\_\-v\-e\-r\-s\-i\-o\-n\-\_\-\_\- & \textbf{Value:} 
{\tt '\-\$\-I\-d\-:\-~\-p\-y\-d\-s\-p\-.\-p\-y\-~\-4\-0\-1\-~\-2\-0\-0\-6\--\-0\-7\--\-1\-1\-~\-2\-2\-:\-3\-1\-:\-5\-8\-Z\-~\-d\-r\-e\-w\-~\-\$\-'\-}&\\
\cline{1-2}
\raggedright \_\-\_\-a\-u\-t\-h\-o\-r\-\_\-\_\- & \textbf{Value:} 
{\tt '\-\$\-A\-u\-t\-h\-o\-r\-:\-~\-d\-r\-e\-w\-~\-\$\-'\-}&\\
\cline{1-2}
\raggedright \_\-\_\-U\-R\-L\-\_\-\_\- & \textbf{Value:} 
{\tt '\-\$\-U\-R\-L\-:\-~\-h\-t\-t\-p\-:\-/\-/\-a\-s\-t\-r\-o\-.\-p\-a\-s\-.\-r\-o\-c\-h\-e\-s\-t\-e\-r\-.\-e\-d\-u\-/\-s\-v\-n\-/\-p\-y\-d\-s\-p\-/\-t\-r\-u\-n\-k\-/\-p\-y\-d\-s\-p\-/\-p\-y\-d\-s\-p\-.\-p\-y\-~\-\$\-'\-}&\\
\cline{1-2}
\raggedright t\-m\-p\-s\- & \textbf{Value:} 
{\tt {\textless}\-f\-u\-n\-c\-t\-i\-o\-n\-~\-t\-m\-p\-s\-~\-a\-t\-~\-0\-x\-f\-6\-c\-2\-7\-7\-2\-c\-{\textgreater}\-}&\\
\cline{1-2}
\raggedright p\-a\-g\-e\-t\-o\-p\- & \textbf{Value:} 
{\tt '\-~\-{\textbackslash}\-n\-{\textless}\-h\-t\-m\-l\-{\textgreater}\-{\textbackslash}\-n\-{\textless}\-h\-e\-a\-d\-{\textgreater}\-{\textbackslash}\-n\-{\textless}\-t\-i\-t\-l\-e\-{\textgreater}\-P\-y\-d\-s\-p\-~\-o\-n\-~\-I\-t\-c\-h\-y\-{\textless}\-/\-t\-i\-t\-l\-e\-{\textgreater}\-{\textbackslash}\-n\-{\textless}\-/\-h\-e\-a\-d\-{\textgreater}\-{\textbackslash}\-n\-{\textless}\-b\-o\-d\-y\-{\textgreater}\-{\textbackslash}\-n\-{\textless}\-h\-1\-{\textgreater}\-P\-y\-d\-s\-p\-.\-.\-.\-}&\\
\cline{1-2}
\raggedright p\-a\-g\-e\-e\-n\-d\- & \textbf{Value:} 
{\tt '\-{\textbackslash}\-n\-{\textless}\-/\-b\-o\-d\-y\-{\textgreater}\-{\textbackslash}\-n\-{\textless}\-/\-h\-t\-m\-l\-{\textgreater}\-{\textbackslash}\-n\-'\-}&\\
\cline{1-2}
\raggedright i\-f\-u\-n\-c\-s\- & \textbf{Value:} 
{\tt [\-]\-}&\\
\cline{1-2}
\end{longtable}


%%%%%%%%%%%%%%%%%%%%%%%%%%%%%%%%%%%%%%%%%%%%%%%%%%%%%%%%%%%%%%%%%%%%%%%%%%%
%%                           Class Description                           %%
%%%%%%%%%%%%%%%%%%%%%%%%%%%%%%%%%%%%%%%%%%%%%%%%%%%%%%%%%%%%%%%%%%%%%%%%%%%

    \index{pydsp \textit{(module)}!pydsp.pydspWebServer \textit{(class)}|(}
\subsection{Class pydspWebServer}

    \label{pydsp:pydspWebServer}
\begin{tabular}{cccccccccc}
% Line for SocketServer.BaseRequestHandler, linespec=[False, False, False]
\multicolumn{2}{r}{\settowidth{\BCL}{SocketServer.BaseRequestHandler}\multirow{2}{\BCL}{SocketServer.BaseRequestHandler}}
&&
&&
&&
  \\\cline{3-3}
  &&\multicolumn{1}{c|}{}
&&
&&
&&
  \\
% Line for SocketServer.StreamRequestHandler, linespec=[False, False]
\multicolumn{4}{r}{\settowidth{\BCL}{SocketServer.StreamRequestHandler}\multirow{2}{\BCL}{SocketServer.StreamRequestHandler}}
&&
&&
  \\\cline{5-5}
  &&&&\multicolumn{1}{c|}{}
&&
&&
  \\
% Line for BaseHTTPServer.BaseHTTPRequestHandler, linespec=[False]
\multicolumn{6}{r}{\settowidth{\BCL}{BaseHTTPServer.BaseHTTPRequestHandler}\multirow{2}{\BCL}{BaseHTTPServer.BaseHTTPRequestHandler}}
&&
  \\\cline{7-7}
  &&&&&&\multicolumn{1}{c|}{}
&&
  \\
&&&&&&\multicolumn{2}{l}{\textbf{pydsp.pydspWebServer}}
\end{tabular}

The class that defines the internals for the pydsp web server.


%%%%%%%%%%%%%%%%%%%%%%%%%%%%%%%%%%%%%%%%%%%%%%%%%%%%%%%%%%%%%%%%%%%%%%%%%%%
%%                                Methods                                %%
%%%%%%%%%%%%%%%%%%%%%%%%%%%%%%%%%%%%%%%%%%%%%%%%%%%%%%%%%%%%%%%%%%%%%%%%%%%

  \subsubsection{Methods}

    \label{pydsp:pydspWebServer:do_GET}
    \index{pydsp \textit{(module)}!pydsp.pydspWebServer \textit{(class)}!pydsp.pydspWebServer.do\_GET \textit{(method)}}

    \vspace{0.5ex}

    \begin{boxedminipage}{\textwidth}

    \raggedright \textbf{do\_GET}(\textit{self})

    \end{boxedminipage}

    \label{pydsp:pydspWebServer:do_POST}
    \index{pydsp \textit{(module)}!pydsp.pydspWebServer \textit{(class)}!pydsp.pydspWebServer.do\_POST \textit{(method)}}

    \vspace{0.5ex}

    \begin{boxedminipage}{\textwidth}

    \raggedright \textbf{do\_POST}(\textit{self})

    \end{boxedminipage}

    \vspace{0.5ex}

    \begin{boxedminipage}{\textwidth}

    \raggedright \textbf{log\_request}(\textit{self}, *\textit{args})

    Log an accepted request.

    This is called by send\_reponse().

    \vspace{1ex}

      Overrides: BaseHTTPServer.BaseHTTPRequestHandler.log\_request 	extit{(inherited documentation)}

    \end{boxedminipage}

    \vspace{0.5ex}

    \begin{boxedminipage}{\textwidth}

    \raggedright \textbf{log\_message}(\textit{self}, *\textit{args})

    Log an accepted request.

    This is called by send\_reponse().

    \vspace{1ex}

      Overrides: BaseHTTPServer.BaseHTTPRequestHandler.log\_message 	extit{(inherited documentation)}

    \end{boxedminipage}

    \label{pydsp:pydspWebServer:do_GET}
    \index{pydsp \textit{(module)}!pydsp.pydspWebServer \textit{(class)}!pydsp.pydspWebServer.do\_GET \textit{(method)}}

    \vspace{0.5ex}

    \begin{boxedminipage}{\textwidth}

    \raggedright \textbf{do\_HEAD}(\textit{self})

    \end{boxedminipage}

    \label{SocketServer:BaseRequestHandler:__init__}
    \index{SocketServer.BaseRequestHandler.\_\_init\_\_ \textit{(function)}}

    \vspace{0.5ex}

    \begin{boxedminipage}{\textwidth}

    \raggedright \textbf{\_\_init\_\_}(\textit{self}, \textit{request}, \textit{client\_address}, \textit{server})

    \end{boxedminipage}

    \label{BaseHTTPServer:BaseHTTPRequestHandler:address_string}
    \index{BaseHTTPServer.BaseHTTPRequestHandler.address\_string \textit{(function)}}

    \vspace{0.5ex}

    \begin{boxedminipage}{\textwidth}

    \raggedright \textbf{address\_string}(\textit{self})

    \vspace{-1.5ex}

    \rule{\textwidth}{0.5\fboxrule}
    Return the client address formatted for logging.

    This version looks up the full hostname using gethostbyaddr(), and 
    tries to find a name that contains at least one dot.

    \vspace{1ex}

    \end{boxedminipage}

    \label{BaseHTTPServer:BaseHTTPRequestHandler:date_time_string}
    \index{BaseHTTPServer.BaseHTTPRequestHandler.date\_time\_string \textit{(function)}}

    \vspace{0.5ex}

    \begin{boxedminipage}{\textwidth}

    \raggedright \textbf{date\_time\_string}(\textit{self})

    \vspace{-1.5ex}

    \rule{\textwidth}{0.5\fboxrule}
    Return the current date and time formatted for a message header.

    \vspace{1ex}

    \end{boxedminipage}

    \label{BaseHTTPServer:BaseHTTPRequestHandler:end_headers}
    \index{BaseHTTPServer.BaseHTTPRequestHandler.end\_headers \textit{(function)}}

    \vspace{0.5ex}

    \begin{boxedminipage}{\textwidth}

    \raggedright \textbf{end\_headers}(\textit{self})

    \vspace{-1.5ex}

    \rule{\textwidth}{0.5\fboxrule}
    Send the blank line ending the MIME headers.

    \vspace{1ex}

    \end{boxedminipage}

    \vspace{0.5ex}

    \begin{boxedminipage}{\textwidth}

    \raggedright \textbf{finish}(\textit{self})

      Overrides: SocketServer.BaseRequestHandler.finish

    \end{boxedminipage}

    \vspace{0.5ex}

    \begin{boxedminipage}{\textwidth}

    \raggedright \textbf{handle}(\textit{self})

    \vspace{-1.5ex}

    \rule{\textwidth}{0.5\fboxrule}
    Handle multiple requests if necessary.

    \vspace{1ex}

      Overrides: SocketServer.BaseRequestHandler.handle

    \end{boxedminipage}

    \label{BaseHTTPServer:BaseHTTPRequestHandler:handle_one_request}
    \index{BaseHTTPServer.BaseHTTPRequestHandler.handle\_one\_request \textit{(function)}}

    \vspace{0.5ex}

    \begin{boxedminipage}{\textwidth}

    \raggedright \textbf{handle\_one\_request}(\textit{self})

    \vspace{-1.5ex}

    \rule{\textwidth}{0.5\fboxrule}
    Handle a single HTTP request.

    You normally don't need to override this method; see the class 
    \_\_doc\_\_ string for information on how to handle specific HTTP 
    commands such as GET and POST.

    \vspace{1ex}

    \end{boxedminipage}

    \label{BaseHTTPServer:BaseHTTPRequestHandler:log_date_time_string}
    \index{BaseHTTPServer.BaseHTTPRequestHandler.log\_date\_time\_string \textit{(function)}}

    \vspace{0.5ex}

    \begin{boxedminipage}{\textwidth}

    \raggedright \textbf{log\_date\_time\_string}(\textit{self})

    \vspace{-1.5ex}

    \rule{\textwidth}{0.5\fboxrule}
    Return the current time formatted for logging.

    \vspace{1ex}

    \end{boxedminipage}

    \label{BaseHTTPServer:BaseHTTPRequestHandler:log_error}
    \index{BaseHTTPServer.BaseHTTPRequestHandler.log\_error \textit{(function)}}

    \vspace{0.5ex}

    \begin{boxedminipage}{\textwidth}

    \raggedright \textbf{log\_error}(\textit{self}, *\textit{args})

    \vspace{-1.5ex}

    \rule{\textwidth}{0.5\fboxrule}
    Log an error.

    This is called when a request cannot be fulfilled.  By default it 
    passes the message on to log\_message().

    Arguments are the same as for log\_message().

    XXX This should go to the separate error log.

    \vspace{1ex}

    \end{boxedminipage}

    \label{BaseHTTPServer:BaseHTTPRequestHandler:parse_request}
    \index{BaseHTTPServer.BaseHTTPRequestHandler.parse\_request \textit{(function)}}

    \vspace{0.5ex}

    \begin{boxedminipage}{\textwidth}

    \raggedright \textbf{parse\_request}(\textit{self})

    \vspace{-1.5ex}

    \rule{\textwidth}{0.5\fboxrule}
    Parse a request (internal).

    The request should be stored in self.raw\_requestline; the results are 
    in self.command, self.path, self.request\_version and self.headers.

    Return True for success, False for failure; on failure, an error is 
    sent back.

    \vspace{1ex}

    \end{boxedminipage}

    \label{BaseHTTPServer:BaseHTTPRequestHandler:send_error}
    \index{BaseHTTPServer.BaseHTTPRequestHandler.send\_error \textit{(function)}}

    \vspace{0.5ex}

    \begin{boxedminipage}{\textwidth}

    \raggedright \textbf{send\_error}(\textit{self}, \textit{code}, \textit{message}=\texttt{None})

    \vspace{-1.5ex}

    \rule{\textwidth}{0.5\fboxrule}
    Send and log an error reply.

    Arguments are the error code, and a detailed message. The detailed 
    message defaults to the short entry matching the response code.

    This sends an error response (so it must be called before any output 
    has been generated), logs the error, and finally sends a piece of HTML 
    explaining the error to the user.

    \vspace{1ex}

    \end{boxedminipage}

    \label{BaseHTTPServer:BaseHTTPRequestHandler:send_header}
    \index{BaseHTTPServer.BaseHTTPRequestHandler.send\_header \textit{(function)}}

    \vspace{0.5ex}

    \begin{boxedminipage}{\textwidth}

    \raggedright \textbf{send\_header}(\textit{self}, \textit{keyword}, \textit{value})

    \vspace{-1.5ex}

    \rule{\textwidth}{0.5\fboxrule}
    Send a MIME header.

    \vspace{1ex}

    \end{boxedminipage}

    \label{BaseHTTPServer:BaseHTTPRequestHandler:send_response}
    \index{BaseHTTPServer.BaseHTTPRequestHandler.send\_response \textit{(function)}}

    \vspace{0.5ex}

    \begin{boxedminipage}{\textwidth}

    \raggedright \textbf{send\_response}(\textit{self}, \textit{code}, \textit{message}=\texttt{None})

    \vspace{-1.5ex}

    \rule{\textwidth}{0.5\fboxrule}
    Send the response header and log the response code.

    Also send two standard headers with the server software version and the
    current date.

    \vspace{1ex}

    \end{boxedminipage}

    \vspace{0.5ex}

    \begin{boxedminipage}{\textwidth}

    \raggedright \textbf{setup}(\textit{self})

      Overrides: SocketServer.BaseRequestHandler.setup

    \end{boxedminipage}

    \label{BaseHTTPServer:BaseHTTPRequestHandler:version_string}
    \index{BaseHTTPServer.BaseHTTPRequestHandler.version\_string \textit{(function)}}

    \vspace{0.5ex}

    \begin{boxedminipage}{\textwidth}

    \raggedright \textbf{version\_string}(\textit{self})

    \vspace{-1.5ex}

    \rule{\textwidth}{0.5\fboxrule}
    Return the server software version string.

    \vspace{1ex}

    \end{boxedminipage}


%%%%%%%%%%%%%%%%%%%%%%%%%%%%%%%%%%%%%%%%%%%%%%%%%%%%%%%%%%%%%%%%%%%%%%%%%%%
%%                            Class Variables                            %%
%%%%%%%%%%%%%%%%%%%%%%%%%%%%%%%%%%%%%%%%%%%%%%%%%%%%%%%%%%%%%%%%%%%%%%%%%%%

  \subsubsection{Class Variables}

\begin{longtable}{|p{.30\textwidth}|p{.62\textwidth}|l}
\cline{1-2}
\cline{1-2} \centering \textbf{Name} & \centering \textbf{Description}& \\
\cline{1-2}
\endhead\cline{1-2}\multicolumn{3}{r}{\small\textit{continued on next page}}\\\endfoot\cline{1-2}
\endlastfoot\raggedright s\-e\-r\-v\-e\-r\-\_\-v\-e\-r\-s\-i\-o\-n\- & \textbf{Value:} 
{\tt '\-p\-y\-d\-s\-p\-H\-T\-T\-P\-/\-1\-.\-0\-'\-}&\\
\cline{1-2}
\raggedright e\-r\-r\-o\-r\-\_\-m\-e\-s\-s\-a\-g\-e\-\_\-f\-o\-r\-m\-a\-t\- & \textbf{Value:} 
{\tt '\-{\textless}\-h\-e\-a\-d\-{\textgreater}\-{\textbackslash}\-n\-{\textless}\-t\-i\-t\-l\-e\-{\textgreater}\-E\-r\-r\-o\-r\-~\-r\-e\-s\-p\-o\-n\-s\-e\-{\textless}\-/\-t\-i\-t\-l\-e\-{\textgreater}\-{\textbackslash}\-n\-{\textless}\-/\-h\-e\-a\-d\-{\textgreater}\-{\textbackslash}\-n\-{\textless}\-b\-o\-d\-y\-{\textgreater}\-{\textbackslash}\-n\-{\textless}\-h\-1\-{\textgreater}\-E\-r\-r\-o\-r\-~\-r\-e\-s\-p\-o\-n\-s\-e\-{\textless}\-/\-.\-.\-.\-}&\\
\cline{1-2}
\raggedright m\-o\-n\-t\-h\-n\-a\-m\-e\- & \textbf{Value:} 
{\tt [\-N\-o\-n\-e\-,\-~\-'\-J\-a\-n\-'\-,\-~\-'\-F\-e\-b\-'\-,\-~\-'\-M\-a\-r\-'\-,\-~\-'\-A\-p\-r\-'\-,\-~\-'\-M\-a\-y\-'\-,\-~\-'\-J\-u\-n\-'\-,\-~\-'\-J\-u\-l\-'\-,\-~\-'\-A\-u\-g\-'\-,\-~\-'\-S\-e\-p\-'\-,\-~\-'\-O\-c\-t\-'\-,\-~\-.\-.\-.\-}&\\
\cline{1-2}
\raggedright p\-r\-o\-t\-o\-c\-o\-l\-\_\-v\-e\-r\-s\-i\-o\-n\- & \textbf{Value:} 
{\tt '\-H\-T\-T\-P\-/\-1\-.\-0\-'\-}&\\
\cline{1-2}
\raggedright r\-b\-u\-f\-s\-i\-z\-e\- & \textbf{Value:} 
{\tt -\-1\-}&\\
\cline{1-2}
\raggedright r\-e\-s\-p\-o\-n\-s\-e\-s\- & \textbf{Value:} 
{\tt \{\-4\-0\-0\-:\-~\-(\-'\-B\-a\-d\-~\-r\-e\-q\-u\-e\-s\-t\-'\-,\-~\-'\-B\-a\-d\-~\-r\-e\-q\-u\-e\-s\-t\-~\-s\-y\-n\-t\-a\-x\-~\-o\-r\-~\-u\-n\-s\-u\-p\-p\-o\-r\-t\-e\-d\-~\-m\-e\-t\-h\-o\-d\-'\-)\-,\-~\-4\-0\-1\-:\-~\-(\-'\-U\-n\-a\-.\-.\-.\-}&\\
\cline{1-2}
\raggedright s\-y\-s\-\_\-v\-e\-r\-s\-i\-o\-n\- & \textbf{Value:} 
{\tt '\-P\-y\-t\-h\-o\-n\-/\-2\-.\-3\-.\-4\-'\-}&\\
\cline{1-2}
\raggedright w\-b\-u\-f\-s\-i\-z\-e\- & \textbf{Value:} 
{\tt 0\-}&\\
\cline{1-2}
\raggedright w\-e\-e\-k\-d\-a\-y\-n\-a\-m\-e\- & \textbf{Value:} 
{\tt [\-'\-M\-o\-n\-'\-,\-~\-'\-T\-u\-e\-'\-,\-~\-'\-W\-e\-d\-'\-,\-~\-'\-T\-h\-u\-'\-,\-~\-'\-F\-r\-i\-'\-,\-~\-'\-S\-a\-t\-'\-,\-~\-'\-S\-u\-n\-'\-]\-}&\\
\cline{1-2}
\end{longtable}

    \index{pydsp \textit{(module)}!pydsp.pydspWebServer \textit{(class)}|)}
    \index{pydsp \textit{(module)}|)}
