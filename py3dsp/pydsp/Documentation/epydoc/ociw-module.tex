%
% API Documentation for API Documentation
% Module ociw
%
% Generated by epydoc 3.0alpha2
% [Thu Jul 20 16:44:54 2006]
%

%%%%%%%%%%%%%%%%%%%%%%%%%%%%%%%%%%%%%%%%%%%%%%%%%%%%%%%%%%%%%%%%%%%%%%%%%%%
%%                          Module Description                           %%
%%%%%%%%%%%%%%%%%%%%%%%%%%%%%%%%%%%%%%%%%%%%%%%%%%%%%%%%%%%%%%%%%%%%%%%%%%%

    \index{ociw \textit{(module)}|(}
\section{Module ociw}

    \label{ociw}
ociw.py: lowest-level interface to the device driver and the running DSP.

no real dependencies, but some of it assumes a certain data protocol 
supported by the running dsp program.

It also assumes a single PCI card. A more generic approach would create an 
object for each card. The driver would need a revisit in this situation.

this module, and sload.py, replaced the ociw.so Python extension module 
that was written in C.


%%%%%%%%%%%%%%%%%%%%%%%%%%%%%%%%%%%%%%%%%%%%%%%%%%%%%%%%%%%%%%%%%%%%%%%%%%%
%%                               Functions                               %%
%%%%%%%%%%%%%%%%%%%%%%%%%%%%%%%%%%%%%%%%%%%%%%%%%%%%%%%%%%%%%%%%%%%%%%%%%%%

  \subsection{Functions}

    \label{ociw:open}
    \index{ociw \textit{(module)}!ociw.open \textit{(function)}}

    \vspace{0.5ex}

    \begin{boxedminipage}{\textwidth}

    \raggedright \textbf{open}(\textit{devname}=\texttt{"/dev/ociw0"})

    \vspace{-1.5ex}

    \rule{\textwidth}{0.5\fboxrule}
    Open the device. (perhaps a bad name, Python has a built-in named 
    open.)

    \vspace{1ex}

    \end{boxedminipage}

    \label{ociw:write}
    \index{ociw \textit{(module)}!ociw.write \textit{(function)}}

    \vspace{0.5ex}

    \begin{boxedminipage}{\textwidth}

    \raggedright \textbf{write}(\textit{i})

    \vspace{-1.5ex}

    \rule{\textwidth}{0.5\fboxrule}
    Write one 16 bit word to the device.

    \vspace{1ex}

    \end{boxedminipage}

    \label{ociw:read}
    \index{ociw \textit{(module)}!ociw.read \textit{(function)}}

    \vspace{0.5ex}

    \begin{boxedminipage}{\textwidth}

    \raggedright \textbf{read}(\textit{expected}=\texttt{True})

    \vspace{-1.5ex}

    \rule{\textwidth}{0.5\fboxrule}
    Read one 16 bit word from the device.

    \vspace{1ex}

    \end{boxedminipage}

    \label{ociw:reset}
    \index{ociw \textit{(module)}!ociw.reset \textit{(function)}}

    \vspace{0.5ex}

    \begin{boxedminipage}{\textwidth}

    \raggedright \textbf{reset}()

    \vspace{-1.5ex}

    \rule{\textwidth}{0.5\fboxrule}
    Reset the device.

    \vspace{1ex}

    \end{boxedminipage}

    \label{ociw:command}
    \index{ociw \textit{(module)}!ociw.command \textit{(function)}}

    \vspace{0.5ex}

    \begin{boxedminipage}{\textwidth}

    \raggedright \textbf{command}(\textit{cmd}, \textit{addr})

    \vspace{-1.5ex}

    \rule{\textwidth}{0.5\fboxrule}
    Write 8 bit command and 24 bit address to the clock program. wasteful 
    for commands with only 8 bits of data.

    \vspace{1ex}

    \end{boxedminipage}

    \label{ociw:data24}
    \index{ociw \textit{(module)}!ociw.data24 \textit{(function)}}

    \vspace{0.5ex}

    \begin{boxedminipage}{\textwidth}

    \raggedright \textbf{data24}(\textit{data})

    \vspace{-1.5ex}

    \rule{\textwidth}{0.5\fboxrule}
    Write a 24 bit data word to the clock program.

    \vspace{1ex}

    \end{boxedminipage}

    \label{ociw:clear_fifo}
    \index{ociw \textit{(module)}!ociw.clear\_fifo \textit{(function)}}

    \vspace{0.5ex}

    \begin{boxedminipage}{\textwidth}

    \raggedright \textbf{clear\_fifo}()

    \vspace{-1.5ex}

    \rule{\textwidth}{0.5\fboxrule}
    make sure nothing is in the fifo

    \vspace{1ex}

    \end{boxedminipage}

    \label{ociw:peek_fifo}
    \index{ociw \textit{(module)}!ociw.peek\_fifo \textit{(function)}}

    \vspace{0.5ex}

    \begin{boxedminipage}{\textwidth}

    \raggedright \textbf{peek\_fifo}()

    \vspace{-1.5ex}

    \rule{\textwidth}{0.5\fboxrule}
    see how much is in the fifo, without reading it.

    \vspace{1ex}

    \end{boxedminipage}

    \label{ociw:interrupted}
    \index{ociw \textit{(module)}!ociw.interrupted \textit{(function)}}

    \vspace{0.5ex}

    \begin{boxedminipage}{\textwidth}

    \raggedright \textbf{interrupted}()

    \vspace{-1.5ex}

    \rule{\textwidth}{0.5\fboxrule}
    RawFrame objects call interrupted in all data acquisition loops to see 
    if an interrupt has occurred. This is how acquisition is aborted.

    The default implementation does nothing.

    Override this function with something that actually works.

    \vspace{1ex}

    \end{boxedminipage}


%%%%%%%%%%%%%%%%%%%%%%%%%%%%%%%%%%%%%%%%%%%%%%%%%%%%%%%%%%%%%%%%%%%%%%%%%%%
%%                               Variables                               %%
%%%%%%%%%%%%%%%%%%%%%%%%%%%%%%%%%%%%%%%%%%%%%%%%%%%%%%%%%%%%%%%%%%%%%%%%%%%

  \subsection{Variables}

\begin{longtable}{|p{.30\textwidth}|p{.62\textwidth}|l}
\cline{1-2}
\cline{1-2} \centering \textbf{Name} & \centering \textbf{Description}& \\
\cline{1-2}
\endhead\cline{1-2}\multicolumn{3}{r}{\small\textit{continued on next page}}\\\endfoot\cline{1-2}
\endlastfoot\raggedright \_\-\_\-r\-e\-v\-i\-s\-i\-o\-n\-\_\-\_\- & \textbf{Value:} 
{\tt '\-\$\-I\-d\-:\-\$\-'\-}&\\
\cline{1-2}
\raggedright f\-d\- & \textbf{Value:} 
{\tt N\-o\-n\-e\-}&\\
\cline{1-2}
\raggedright d\-e\-v\- & \textbf{Value:} 
{\tt N\-o\-n\-e\-}&\\
\cline{1-2}
\raggedright S\-E\-T\-\_\-D\-A\-T\-A\-\_\-M\-S\- & \textbf{Value:} 
{\tt 1\-}&\\
\cline{1-2}
\raggedright S\-E\-T\-\_\-D\-A\-T\-A\-\_\-N\-S\- & \textbf{Value:} 
{\tt 2\-}&\\
\cline{1-2}
\raggedright S\-E\-T\-\_\-D\-A\-T\-A\-\_\-L\-S\- & \textbf{Value:} 
{\tt 3\-}&\\
\cline{1-2}
\raggedright S\-E\-T\-\_\-A\-D\-D\-R\-\_\-M\-S\- & \textbf{Value:} 
{\tt 4\-}&\\
\cline{1-2}
\raggedright S\-E\-T\-\_\-A\-D\-D\-R\-\_\-N\-S\- & \textbf{Value:} 
{\tt 5\-}&\\
\cline{1-2}
\raggedright S\-E\-T\-\_\-A\-D\-D\-R\-\_\-L\-S\- & \textbf{Value:} 
{\tt 6\-}&\\
\cline{1-2}
\end{longtable}


%%%%%%%%%%%%%%%%%%%%%%%%%%%%%%%%%%%%%%%%%%%%%%%%%%%%%%%%%%%%%%%%%%%%%%%%%%%
%%                           Class Description                           %%
%%%%%%%%%%%%%%%%%%%%%%%%%%%%%%%%%%%%%%%%%%%%%%%%%%%%%%%%%%%%%%%%%%%%%%%%%%%

    \index{ociw \textit{(module)}!ociw.AbstractFrame \textit{(class)}|(}
\subsection{Class AbstractFrame}

    \label{ociw:AbstractFrame}
unused design concept. left here as food for thought.


%%%%%%%%%%%%%%%%%%%%%%%%%%%%%%%%%%%%%%%%%%%%%%%%%%%%%%%%%%%%%%%%%%%%%%%%%%%
%%                                Methods                                %%
%%%%%%%%%%%%%%%%%%%%%%%%%%%%%%%%%%%%%%%%%%%%%%%%%%%%%%%%%%%%%%%%%%%%%%%%%%%

  \subsubsection{Methods}

    \label{ociw:AbstractFrame:__init__}
    \index{ociw \textit{(module)}!ociw.AbstractFrame \textit{(class)}!ociw.AbstractFrame.\_\_init\_\_ \textit{(method)}}

    \vspace{0.5ex}

    \begin{boxedminipage}{\textwidth}

    \raggedright \textbf{\_\_init\_\_}(\textit{self}, **\textit{kwds})

    \end{boxedminipage}

    \label{ociw:AbstractFrame:grab}
    \index{ociw \textit{(module)}!ociw.AbstractFrame \textit{(class)}!ociw.AbstractFrame.grab \textit{(method)}}

    \vspace{0.5ex}

    \begin{boxedminipage}{\textwidth}

    \raggedright \textbf{grab}(\textit{self})

    \end{boxedminipage}

    \index{ociw \textit{(module)}!ociw.AbstractFrame \textit{(class)}|)}

%%%%%%%%%%%%%%%%%%%%%%%%%%%%%%%%%%%%%%%%%%%%%%%%%%%%%%%%%%%%%%%%%%%%%%%%%%%
%%                           Class Description                           %%
%%%%%%%%%%%%%%%%%%%%%%%%%%%%%%%%%%%%%%%%%%%%%%%%%%%%%%%%%%%%%%%%%%%%%%%%%%%

    \index{ociw \textit{(module)}!ociw.RawFrame \textit{(class)}|(}
\subsection{Class RawFrame}

    \label{ociw:RawFrame}
Image data acquisition object. Acquires a single frame. Instantiate it with
nrows, ncols, prepix, and postpix. possibly add special funcs to the base 
class. have the object acquire by calling its grab function. it returns the
image data in a string. tag it with metadata. The public interface is the 
grab method. Interested objects can subscribe to events before and after 
the data has come in.


%%%%%%%%%%%%%%%%%%%%%%%%%%%%%%%%%%%%%%%%%%%%%%%%%%%%%%%%%%%%%%%%%%%%%%%%%%%
%%                                Methods                                %%
%%%%%%%%%%%%%%%%%%%%%%%%%%%%%%%%%%%%%%%%%%%%%%%%%%%%%%%%%%%%%%%%%%%%%%%%%%%

  \subsubsection{Methods}

    \label{ociw:RawFrame:__init__}
    \index{ociw \textit{(module)}!ociw.RawFrame \textit{(class)}!ociw.RawFrame.\_\_init\_\_ \textit{(method)}}

    \vspace{0.5ex}

    \begin{boxedminipage}{\textwidth}

    \raggedright \textbf{\_\_init\_\_}(\textit{self}, \textit{nrows}, \textit{ncols}, \textit{prepix}=\texttt{0}, \textit{postpix}=\texttt{0})

    \vspace{-1.5ex}

    \rule{\textwidth}{0.5\fboxrule}
    create a new acquisition object. each frame has a size expressed in 
    rows and columns, and a few pixels that always precede or follow the 
    main frame.

    Each instance makes a copy of the base class' list of subscribers and 
    allows the user to add new functions to this list or perhaps delete the
    ones that are there.

    \vspace{1ex}

    \end{boxedminipage}

    \label{ociw:RawFrame:preframeAlert}
    \index{ociw \textit{(module)}!ociw.RawFrame \textit{(class)}!ociw.RawFrame.preframeAlert \textit{(method)}}

    \vspace{0.5ex}

    \begin{boxedminipage}{\textwidth}

    \raggedright \textbf{preframeAlert}(\textit{self})

    \vspace{-1.5ex}

    \rule{\textwidth}{0.5\fboxrule}
    Publish-subscribe hook. Called BEFORE frame is started.

    \vspace{1ex}

    \end{boxedminipage}

    \label{ociw:RawFrame:frameAlert}
    \index{ociw \textit{(module)}!ociw.RawFrame \textit{(class)}!ociw.RawFrame.frameAlert \textit{(method)}}

    \vspace{0.5ex}

    \begin{boxedminipage}{\textwidth}

    \raggedright \textbf{frameAlert}(\textit{self})

    \vspace{-1.5ex}

    \rule{\textwidth}{0.5\fboxrule}
    Publish-subscribe hook. Called AFTER frame is successfully in.

    \vspace{1ex}

    \end{boxedminipage}

    \label{ociw:RawFrame:grab_predata}
    \index{ociw \textit{(module)}!ociw.RawFrame \textit{(class)}!ociw.RawFrame.grab\_predata \textit{(method)}}

    \vspace{0.5ex}

    \begin{boxedminipage}{\textwidth}

    \raggedright \textbf{grab\_predata}(\textit{self})

    \vspace{-1.5ex}

    \rule{\textwidth}{0.5\fboxrule}
    Acquire the preceding pixels for the image frame.

    \vspace{1ex}

    \end{boxedminipage}

    \label{ociw:RawFrame:grab_data}
    \index{ociw \textit{(module)}!ociw.RawFrame \textit{(class)}!ociw.RawFrame.grab\_data \textit{(method)}}

    \vspace{0.5ex}

    \begin{boxedminipage}{\textwidth}

    \raggedright \textbf{grab\_data}(\textit{self})

    \vspace{-1.5ex}

    \rule{\textwidth}{0.5\fboxrule}
    Acquire the main data for the frame.

    \vspace{1ex}

    \end{boxedminipage}

    \label{ociw:RawFrame:grab_postdata}
    \index{ociw \textit{(module)}!ociw.RawFrame \textit{(class)}!ociw.RawFrame.grab\_postdata \textit{(method)}}

    \vspace{0.5ex}

    \begin{boxedminipage}{\textwidth}

    \raggedright \textbf{grab\_postdata}(\textit{self})

    \vspace{-1.5ex}

    \rule{\textwidth}{0.5\fboxrule}
    Acquire the trailing pixels for the image frame.

    \vspace{1ex}

    \end{boxedminipage}

    \label{ociw:RawFrame:grab}
    \index{ociw \textit{(module)}!ociw.RawFrame \textit{(class)}!ociw.RawFrame.grab \textit{(method)}}

    \vspace{0.5ex}

    \begin{boxedminipage}{\textwidth}

    \raggedright \textbf{grab}(\textit{self})

    \vspace{-1.5ex}

    \rule{\textwidth}{0.5\fboxrule}
    Acquire the complete frame. Most users should just call this directly.

    \vspace{1ex}

    \end{boxedminipage}


%%%%%%%%%%%%%%%%%%%%%%%%%%%%%%%%%%%%%%%%%%%%%%%%%%%%%%%%%%%%%%%%%%%%%%%%%%%
%%                            Class Variables                            %%
%%%%%%%%%%%%%%%%%%%%%%%%%%%%%%%%%%%%%%%%%%%%%%%%%%%%%%%%%%%%%%%%%%%%%%%%%%%

  \subsubsection{Class Variables}

\begin{longtable}{|p{.30\textwidth}|p{.62\textwidth}|l}
\cline{1-2}
\cline{1-2} \centering \textbf{Name} & \centering \textbf{Description}& \\
\cline{1-2}
\endhead\cline{1-2}\multicolumn{3}{r}{\small\textit{continued on next page}}\\\endfoot\cline{1-2}
\endlastfoot\raggedright p\-r\-e\-f\-u\-n\-c\-s\- & \textbf{Value:} 
{\tt [\-]\-}&\\
\cline{1-2}
\raggedright f\-u\-n\-c\-s\- & \textbf{Value:} 
{\tt [\-]\-}&\\
\cline{1-2}
\end{longtable}

    \index{ociw \textit{(module)}!ociw.RawFrame \textit{(class)}|)}
    \index{ociw \textit{(module)}|)}
